% Chapter 3

\chapter{CHAPTER THREE\\3.0 METHODOLOGY} % Main chapter title

\label{Chapter3} % For referencing the chapter elsewhere, use \ref{Chapter1} 

%-----------------------------------------------------------------------------------

\section{Introduction}
In this chapter, the methods used to investigate our model are expounded, clearly stating the mathematical instruments and constructions, theorems, lemmas and their proofs. 
\subsection{Existence \& Uniqueness Theorem}
\begin{theorem}
	Consider the initial value problem
	$x' = f(x),$  with $x(0) = x_{0}$.
	Suppose that the function $ \textbf{f} $ is continuous and that all its partial derivatives 
	$\dfrac{\delta f_{i}}{\delta x_{j}},\medspace\medspace \emph{i, j = 1, 2,...,n}$ are continuous for $x$ in some open connected set $D\subseteq R_{n}$, then for $x_{0}\in D$, the initial value problem has a solution $x(t)$ on some interval \emph{(-t, t)} about $t = 0$, and the solution is unique. - Nonlinear dynamics and chaos by Steven H. Strogatz \parencite{strogatz2018nonlinear}.
\end{theorem}

\begin{theorem}
	$ V (x) $ is said to be positive (negative) definite in a neighborhood $ U $ of the origin if  $ V (x) > 0 $ $ (V (x) < 0) $ for all $ x  \neq 0 $ in $ U $ , and $ V (0) = 0 $. $ V (x) $
	is positive (negative) semi-definite in a neighborhood U of the origin if $ V (x) \geq 0 $
	$ (V (x) \leq 0) $ for all $ x  \neq 0 $ in $ U $ , and $ V (0) = 0 $.
\end{theorem} 
\begin{theorem}
	Let $ X^{*}(t) = 0 $, $ t \geq t_{0} $ be the zero solution of the regular system $ X^{'} = X (x) $, where $ X (0) = 0 $. Then $ X (x(t)) $ is uniformly stable for $ t \geq t_{0} $ if there exists V (x) with the following properties in some neighborhood of X = 0:
	\begin{itemize}
		\item [i.] $ V (x) $ and its partial derivatives are continuous;
		\item [ii.] $ V (x) $ is positive definite;
		\item [iii.] $ V (x) $ is negative semi-definite.
	\end{itemize}
\end{theorem}
\begin{theorem}
	If we observe all the conditions of the \textbf{Theorem 6}, except the last condition of $ \textbf{iii} $ and instead assume that $ \textbf{iii} $ $ V $ is negative definite. Then the zero solution is asymptotically stable (and such a function $ V $ is called a strong Lyapunov function for the system).
\end{theorem}
\subsection{Basic Reproductive Number $ R_{0} $}
In compartmental models for infectious disease transmission, individual are categorized into two: some are called disease compartments if the individuals therein are infected, while others are called non-disease compartments. Suppose that there are $ n > 0 $ disease compartments and $ m > 0 $ non-disease compartments. Then a general compartmental disease transmission model can be written as 
\begin{align}
x' &= \mathscr{F}(x,y) - \mathscr{V}(x,y),\hspace*{1.5cm} y' = g(x,y) \label{NGM1}
\end{align}
with $ g = (g_{1},\cdots, g_{m})^{T} $. $ x = (x_{1},\cdots, x_{n})^{T} \in \Re_{+}$ and $ y = (y_{1},\cdots, y_{m})^{T} \in \Re_{+} $ represent the populations in the disease and non-disease compartments respectively; $ \mathscr{F} = (\mathscr{F}_{1},\cdots, \mathscr{F}_{n})^{T} $ and $ \mathscr{V} = (\mathscr{V}_{1},\cdots, \mathscr{V}_{n})^{T} $. Then the assumptions as spelt out in \parencite{shuai2013global,van2002reproduction,van2008further} hold and are discussed  and interpreted in detail. Following \parencite{van2002reproduction,van2008further}, define two $ n \times n $ matrices
\begin{align}
F = \left[\dfrac{\partial\mathscr{F}_{i}}{\partial x_{j}}(0,y_{0})\right] \hspace*{0.5cm} \text{ and } \hspace*{0.5cm}  V = \left[\dfrac{\partial\mathscr{V}_{i}}{\partial x_{j}}(0,y_{0})\right]. \label{NGM2}
\end{align} 
Assume that $ F \geq 0 $ and $ V^{-1} \geq 0 $, which are biologically reasonable. Then the next-generation matrix is $ K = FV^{-1} $, and the basic reproductive number $ R_{0} $ can be defined as the spectral radius of K, which is
\begin{align}
R_{0} = \rho(FV^{-1}) \label{NGM3}
\end{align}
To this $ R_{0]} $ there are several measures considered in several literature one of which is the measure 
\begin{align}
R_{0} = \dfrac{s_{DFE}}{s_{EE}} \label{NGM4}
\end{align} 
where $ s_{DFE} $ is the $ s $ value of the DFE and $ s_{EE} $ is the $ s $ value of the EE.\\
\subsection{A Matrix-Theoretic Method}
The matrix-theoretic method is used to prove the statement \parencite{shuai2013global}. It is a systematic method, and it is presented to guide the construction of a Lyapunov function. Taking the same path as \parencite{shuai2013global,castillo2002computation,van2008further}, let us set 
\begin{align}
f(x,y) := (F - V)x - \mathscr{F}(x,y) + \mathscr{V}(x,y) \label{f1}
\end{align}
Then the equation for the disease compartment can be written as 
\begin{align}
x' = (F - V)x - f(x,y) \label{f2}
\end{align}
Let $\psi^{T} \leq 0$ be the left eigenvector of the non-negative matrix $ V^{-1}F $ corresponding to the eigenvalue $ \rho(V^{-1}F) = \rho(FV^{-1}) = R_{0} $ (this is proved in the Appendix). The following result provides a general method to construct a Lyapunov function for (\ref{eq}). \cite{guo2011global,guo2008graph,shuai2011global} used this Lyapunov function involving the Perron eigenvectors to study the global dynamics of the  several specific disease models while \parencite{shuai2013global} used it to consider a general case for infectious diseases. In this paper, we are employing this same method to establish the global stability of our system. 
\begin{theorem}
	Let $ F $, $ V $ and $ f(x,y) $ be defined as in \ref{NGM2} and \ref{f1} respectively. If $ f(x,y) \geq 0 $, in the  $ \Omega \subset \Re_{+}^{n+m} $, $ F \geq 0 $, $ V^{-1} \geq 0 $ and $ R_{0} \leq 1 $, then the function $ \mathscr{D} = \psi^{T} V^{-1}x $ is a Lyapunov function for the system \ref{f2} on $ \Omega $.
\end{theorem}
\textit{Proof.} The proof as followed in \parencite{shuai2013global} gives
\begin{align}
\mathscr{D'} = \mathscr{D}^{'}\mid_{(\ref{eq})} = \psi V^{-1}x' &= \psi V^{-1}(F - V)x -\psi V^{-1}f(x,y)\label{f3}\\
&= (R_{0} - 1)\psi^{T} x - \psi^{T} V^{-1}f(x,y)\label{f4}
\end{align}
Since $ \psi^{T} \geq 0 $, $ V^{-1} \geq $, and $ f(x,y) \geq 0$ in the region $ \Omega $, the last term is non-positive. If $ R_{0} \leq 1 $, then $ \mathscr{D}^{1} \leq 0 $  in $ \Omega $ and thus $ \mathscr{D} $ is a Lyapunov function for the system (\ref{NGM1}).\\
Shuai and Pauline \parencite{shuai2013global} has proven that the Lyapunov function used to prove the global stability of the DFE in $ \Omega $ can also be extended to establish a uniform persistence and thus establish the existence of an EE in $ \Re_{+}^{n+m} $. Find the \textit{theorem 2.2} in \parencite{shuai2013global} and the proof thereof. 
\subsection{Lokta-Volterra Criterion for Lyapunov Functions}
A general form of Lyapunov functions coined from the first integral of the Lokta-Volterra system which is often used in the literature of mathematical biology is used to prove the global stability of the EE. This function takes the form  
\begin{align}
\mathscr{L} &= \Sigma_{i=1}^{n}c_{i}\left(x_{i} - x_{i}^{*} - x_{i}^{*}\ln\dfrac{x_{i}}{x^{*}_{i}}\right)\label{3.11}
\end{align}
where $ x $ are the variables and $ c_{i} $ are carefully selected constants. This criterion has been used many times in establishing the stability or otherwise of many disease models and also present in \cite{shuai2013global}.

