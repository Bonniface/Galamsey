\documentclass[12pt,a4paper]{book}
\usepackage[utf8]{inputenc}
\usepackage[T1]{fontenc}
\usepackage[english]{babel}
\usepackage{amsmath}
\usepackage{amsfonts}
\usepackage{amssymb}
\usepackage{acronym}
\usepackage{mathptmx}
\usepackage{mathpazo}
\usepackage{aligned-overset}
\usepackage{mathrsfs}
\usepackage{makeidx}
\usepackage{graphicx}
\usepackage{booktabs}
\usepackage{lscape}
\usepackage{tabularx}
\usepackage{times}%left=1.4cm, top=.8cm, right=1.4cm, bottom=1.8cm, footskip=.5cm
\usepackage[left=3.75cm, right=2.5cm, top=2.5cm, bottom=2.5cm]{geometry}
\usepackage[section]{placeins}
\newtheorem{theorem}{Theorem}[section]
\newtheorem{dfn}{Definition}[section]
\newtheorem{note}{Note}[section]
\usepackage{titlesec}
\usepackage[style = apa, backend = biber, natbib = true]{biblatex}
\addbibresource{references.bib}
\usepackage{fancyhdr}

\pagestyle{fancy} % Turn on the style
\fancyhf{} % Start with clearing everything in the header and footer
% Set the right side of the footer to be the page number
\fancyfoot[R]{\thepage}

% Redefine plain style, which is used for titlepage and chapter beginnings
% From https://tex.stackexchange.com/a/30230/828
%\fancypagestyle{plain}{%
%	\renewcommand{\headrulewidth}{0pt}%
%	\fancyhf{}%
%	\fancyfoot[R]{\thepage}%
%}
\usepackage{xcolor}
%%%%%%%%%%%%%%%%%%%%%%%%%%%%%%%%%%%%%%%%%%%%%%%%%%%%%%%%%%%%%%%%%%%%%%%%%%%%%%%%%%%%%%%%%%%%%%%%%%%%%%%%
\PassOptionsToPackage{unicode}{hyperref}
\PassOptionsToPackage{hyphens}{url}
%
\usepackage{lmodern}
\usepackage{iftex}
\ifPDFTeX
\usepackage{textcomp} % provide euro and other symbols
\else % if luatex or xetex
\usepackage{unicode-math}
\defaultfontfeatures{Scale=MatchLowercase}
\defaultfontfeatures[\rmfamily]{Ligatures=TeX,Scale=1}
\fi
% Use upquote if available, for straight quotes in verbatim environments
\IfFileExists{upquote.sty}{\usepackage{upquote}}{}
\IfFileExists{microtype.sty}{% use microtype if available
	\usepackage[]{microtype}
	\UseMicrotypeSet[protrusion]{basicmath} % disable protrusion for tt fonts
}{}
\makeatletter
\@ifundefined{KOMAClassName}{% if non-KOMA class
	\IfFileExists{parskip.sty}{%
		\usepackage{parskip}
	}{% else
		\setlength{\parindent}{0pt}
		\setlength{\parskip}{6pt plus 2pt minus 1pt}}
}{% if KOMA class
	\KOMAoptions{parskip=half}}
\makeatother
\usepackage{xcolor}
\newcommand{\VerbBar}{|}
\newcommand{\VERB}{\Verb[commandchars=\\\{\}]}
%\DefineVerbatimEnvironment{Highlighting}{Verbatim}{commandchars=\\\{\}}
% Add ',fontsize=\small' for more characters per line
\usepackage{framed}
\definecolor{shadecolor}{RGB}{248,248,248}
\newenvironment{Shaded}{\begin{snugshade}}{\end{snugshade}}
\newcommand{\AlertTok}[1]{\textcolor[rgb]{0.94,0.16,0.16}{#1}}
\newcommand{\AnnotationTok}[1]{\textcolor[rgb]{0.56,0.35,0.01}{\textbf{\textit{#1}}}}
\newcommand{\AttributeTok}[1]{\textcolor[rgb]{0.77,0.63,0.00}{#1}}
\newcommand{\BaseNTok}[1]{\textcolor[rgb]{0.00,0.00,0.81}{#1}}
\newcommand{\BuiltInTok}[1]{#1}
\newcommand{\CharTok}[1]{\textcolor[rgb]{0.31,0.60,0.02}{#1}}
\newcommand{\CommentTok}[1]{\textcolor[rgb]{0.56,0.35,0.01}{\textit{#1}}}
\newcommand{\CommentVarTok}[1]{\textcolor[rgb]{0.56,0.35,0.01}{\textbf{\textit{#1}}}}
\newcommand{\ConstantTok}[1]{\textcolor[rgb]{0.00,0.00,0.00}{#1}}
\newcommand{\ControlFlowTok}[1]{\textcolor[rgb]{0.13,0.29,0.53}{\textbf{#1}}}
\newcommand{\DataTypeTok}[1]{\textcolor[rgb]{0.13,0.29,0.53}{#1}}
\newcommand{\DecValTok}[1]{\textcolor[rgb]{0.00,0.00,0.81}{#1}}
\newcommand{\DocumentationTok}[1]{\textcolor[rgb]{0.56,0.35,0.01}{\textbf{\textit{#1}}}}
\newcommand{\ErrorTok}[1]{\textcolor[rgb]{0.64,0.00,0.00}{\textbf{#1}}}
\newcommand{\ExtensionTok}[1]{#1}
\newcommand{\FloatTok}[1]{\textcolor[rgb]{0.00,0.00,0.81}{#1}}
\newcommand{\FunctionTok}[1]{\textcolor[rgb]{0.00,0.00,0.00}{#1}}
\newcommand{\ImportTok}[1]{#1}
\newcommand{\InformationTok}[1]{\textcolor[rgb]{0.56,0.35,0.01}{\textbf{\textit{#1}}}}
\newcommand{\KeywordTok}[1]{\textcolor[rgb]{0.13,0.29,0.53}{\textbf{#1}}}
\newcommand{\NormalTok}[1]{#1}
\newcommand{\OperatorTok}[1]{\textcolor[rgb]{0.81,0.36,0.00}{\textbf{#1}}}
\newcommand{\OtherTok}[1]{\textcolor[rgb]{0.56,0.35,0.01}{#1}}
\newcommand{\PreprocessorTok}[1]{\textcolor[rgb]{0.56,0.35,0.01}{\textit{#1}}}
\newcommand{\RegionMarkerTok}[1]{#1}
\newcommand{\SpecialCharTok}[1]{\textcolor[rgb]{0.00,0.00,0.00}{#1}}
\newcommand{\SpecialStringTok}[1]{\textcolor[rgb]{0.31,0.60,0.02}{#1}}
\newcommand{\StringTok}[1]{\textcolor[rgb]{0.31,0.60,0.02}{#1}}
\newcommand{\VariableTok}[1]{\textcolor[rgb]{0.00,0.00,0.00}{#1}}
\newcommand{\VerbatimStringTok}[1]{\textcolor[rgb]{0.31,0.60,0.02}{#1}}
\newcommand{\WarningTok}[1]{\textcolor[rgb]{0.56,0.35,0.01}{\textbf{\textit{#1}}}}
\usepackage{longtable,booktabs,array}
\usepackage{calc} % for calculating minipage widths
% Correct order of tables after \paragraph or \subparagraph
\usepackage{etoolbox}
\makeatletter
\patchcmd\longtable{\par}{\if@noskipsec\mbox{}\fi\par}{}{}
\makeatother
% Allow footnotes in longtable head/foot
\IfFileExists{footnotehyper.sty}{\usepackage{footnotehyper}}{\usepackage{footnote}}
\makesavenoteenv{longtable}
\usepackage{graphicx}
\makeatletter
\def\maxwidth{\ifdim\Gin@nat@width>\linewidth\linewidth\else\Gin@nat@width\fi}
\def\maxheight{\ifdim\Gin@nat@height>\textheight\textheight\else\Gin@nat@height\fi}
\makeatother
% Scale images if necessary, so that they will not overflow the page
% margins by default, and it is still possible to overwrite the defaults
% using explicit options in \includegraphics[width, height, ...]{}
\setkeys{Gin}{width=\maxwidth,height=\maxheight,keepaspectratio}
% Set default figure placement to htbp
\makeatletter
\def\fps@figure{htbp}
\makeatother
\setlength{\emergencystretch}{3em} % prevent overfull lines
\providecommand{\tightlist}{%
	\setlength{\itemsep}{0pt}\setlength{\parskip}{0pt}}
\setcounter{secnumdepth}{-\maxdimen} % remove section numbering
\ifLuaTeX
\usepackage{selnolig}  % disable illegal ligatures
\fi
\IfFileExists{bookmark.sty}{\usepackage{bookmark}}{\usepackage{hyperref}}
\IfFileExists{xurl.sty}{\usepackage{xurl}}{} % add URL line breaks if available
\urlstyle{same} % disable monospaced font for URLs
%%%%%%%%%%%%%%%%%%%%%%%%%%%%%%%%%%%%%%%%%%%%%%%%%%%%%%%%%%%%%%%%%%%%%%%%%%%%%%%%%%%%%%%%%%%%%%%%%%%%%%%%%%
\makeatletter
%\renewcommand{\@makechapterhead}[]{%
	%	\vspace*{20 pt}%
	%	{\setlength{\parindent}{0pt} \raggedright\centring\bf \bfseries\normalfont{\thechapter\ #1\par\nobreak\vspace{20 pt}}}}
\makeatother
\titleformat{\section}{\bfseries\normalfont\centering\bf}{\thesection}{0em}{}
\begin{document}
	\pagestyle{plain}
	\openup 1 em
	\begin{titlepage}
		\pagenumbering{roman}
		%\addcontentsline{toc}{section}{Title Page}
		\begin{center}
			
			\begin{figure}[h]
				\begin{center}
					\includegraphics[width=0.30\linewidth, height=.20\textheight]{images/logo2}
				\end{center}
			\end{figure}
			
			{\normalfont{\textbf{UNIVERSITY OF ENERGY AND NATURAL RESOURCES, SUNYANI}}}\\
			\vspace{1.7cm}
			{\normalfont \textbf{THE VARIABILITY CLIMATE CHANGE IS RESPONSIBLE FOR IN GHANA}}
			
			
			\vspace{2.7cm}
			
			{\normalfont {\textbf{KALONG BONIFACE}}}\\
			{\normalfont {\textbf{FUGAH SELETEY MITCHELL}}}
			\vspace{1.7cm}
			
			
			\begin{center}
				{\normalfont \textbf{DEPARTMENT OF MATHEMATICS AND STATISTIC}\\
					\textbf{SCHOOL OF SCIENCE}}\\
				\vspace{5.3cm}
				
				
				{\normalfont \textbf{\date{\today}}}
			\end{center}
			
		\end{center}
		
		\vfill
	\end{titlepage}
	\pagenumbering{roman}
	\begin{titlepage}
		\begin{center}
			{\normalfont \textbf{THE VARIABILITY CLIMATE CHANGE IS RESPONSIBLE FOR IN GHANA}}
			
			\vspace{3.7cm}
			{\normalfont {by}}\\
			\vspace{1cm}
			{\normalfont {KALONG BONIFACE\hspace\fill  UEB3603118	 \\
					FUGAH SELETEY MITCHELL \hspace\fill UEB3602818 \\ B.Sc. Actuarial Science}}
			\vspace{1.7cm}
			
			\begin{center}
				{\normalfont A Thesis submitted to the Department of Mathematics and Statistics, School of Science, University of Energy and Natural Resources, Sunyani in partial fulfillment of the requirements for the degree of Bachelor of Science in Actuarial Science }\\
				\vspace{2cm}	
				{\normalfont \date{\today}}
			\end{center}
			
		\end{center}
		
		\vfill
	\end{titlepage}	
	\newpage
	{\section*{DECLARATION AND CERTIFICATION}}
	\addcontentsline{toc}{section}{CERTIFICATION}
	We hereby Certified that the thesis entitled “\textbf{THE VARIABILITY CLIMATE CHANGE IS RESPONSIBLE FOR IN  GHANA}”, submitted by
	{\normalfont {\textbf{Kalong Boniface}}} and {\normalfont {\textbf{Fugah Seletey Mitchell}}} to the \textbf{DEPARTMENT} , for the award of  Bachelor’s degree has been accepted by the external examiners and that we have successfully defended the thesis  held today.This dissertation is the result of our own independent investigation and research, we now announce. We are solely liable for any errors in this work even if it has not been submitted anyplace as a long essay or thesis with the intent of awarding a degree.
	\\
%	Kalong Boniface:	................................. \hspace\fill 
%	Date: .................................	 \\
%	Fugah Seletey Mitchell:	................................. \hspace\fill 
%	Date: .................................	 \\
	\begin{center}\textbf{\normalfont \textbf{Supervisor's Certification}}\end{center}
	This study was carried out under the supervisory committee  in accordance with the guidelines on supervisions of graduate studies.\\

	\begin{center}
	\begin{tabular}{lcc}
		\underline{KALONG BONIFACE} &\quad\quad\quad \ldots\ldots\ldots\ldots\ldots\ldots\ldots &\quad\quad\quad \ldots\ldots\ldots\ldots\ldots\ldots \\
		Student &\quad\quad\quad Signature &\quad\quad\quad Date \\
		& & \\
		& & \\
		& & \\
		\underline{FUGAH SELETEY MITCHELL} &\quad\quad\quad \ldots\ldots\ldots\ldots\ldots\ldots\ldots &\quad\quad\quad \ldots\ldots\ldots\ldots\ldots\ldots \\
		Student &\quad\quad\quad Signature &\quad\quad\quad Date \\
		& & \\
		& & \\
		& & \\
		Certified by: & &\\
		\underline{MR. JUSTICE AMENYOR KEESIE} &\quad\quad\quad \ldots\ldots\ldots\ldots\ldots\ldots\ldots &\quad\quad\quad \ldots\ldots\ldots\ldots\ldots\ldots \\
		Supervisor &\quad\quad\quad Signature &\quad\quad\quad Date \\
		& & \\
		& & \\
		& & \\
		Certified by: & &\\
		\underline{DR. ELVIS KWABENA DONKOH} &\quad\quad\quad \ldots\ldots\ldots\ldots\ldots\ldots\ldots &\quad\quad\quad \ldots\ldots\ldots\ldots\ldots\ldots \\
		Head of Department &\quad\quad\quad Signature &\quad\quad\quad Date
	\end{tabular}
\end{center}
	
	
	\newpage
	\section*{\textbf{ABSTRACT}}
	\addcontentsline{toc}{section}{ABSTRACT}
%	All thing change, but how we respond to change is our responsibility, to fare it or embrasse it. Resisting change leads to one fiat. Our own extinction. Time is a smybole of freedom and peace
	In addition to the environmental damage that illicit mining causes, hundreds of young men have died after becoming trapped in unlawful mining holes while looking for minerals.
	The objectives are to (1) quantify, map, and analyze vegetation cover distributions and changes across southern part of Ghana, from 2000 to 2021
	This study aimed to model the effect of climatic variability on vegetation and using vector autoregression (VAR) models. Monthly weather information for 2010 (rainfall, high temperature, and relative humidity)
	
	Data on rainfall from the Ghana Meteorological Agency from 2010 to 2015 and information on malaria from the Ghana Health Service for the same time period. The Granger and immediate causality tests' findings indicated that all factors influence the spread of malaria.
	
	three climate-related factors. The results of the impulse response analyses showed that the months of September, March, and October, respectively, saw the strongest favorable effects of maximum temperature, relative humidity, and rainfall on malaria. As much as 12.65\% of the predicted variance shows a different degree of malaria dependence on meteorological variables.
	
%	 A high degree of variance in vegetation cover for individual dates is explained by HQI at the neighborhood level, although minimal covariability between absolute or relative vegetation cover change and HQI for 2000 was observed.
	\newpage	
	\begin{center}\section*{DEDICATION}\end{center}
	\addcontentsline{toc}{section}{DEDICATION}
	
	Write dedication here   
	\newpage	
	\begin{center}\section*{ACKNOWLEDGMENTS}\end{center}
	\addcontentsline{toc}{section}{ACKNOWLEDGMENTS}
	
	First and foremost, we give praise to the Lord, the Almighty, for his direction, power, and wisdom. Next, we sincerely appreciate our supervisor, Mr. Justice Amenyo Kessie, for his constant support, encouragement, and the time he spent reading this research paper, critiquing it as necessary, explaining the criticism for our understanding, and offering immensely helpful suggestions and recommendations on how to structure this work.
	
	\newpage
	\tableofcontents
	\addcontentsline{toc}{section}{TABLE OF CONTENTS}
	\newpage
	\listoftables
	\addcontentsline{toc}{section}{LIST OF TABLES}
	\newpage
	\listoffigures
	\addcontentsline{toc}{section}{LIST OF FIGURES}
	\newpage
	\addcontentsline{toc}{section}{LIST OF ABBREVIATION}
	\section*{List of Abbreviation}
	\begin{acronym}
		\acro{AR}{Autoregression}  
		\acro{MA}{Moving Average}  
		\acro{ARMA }{Autoregresive Moving Average } 
		\acro{ARIMA}{Autoregressive Integrated Moving Average}  
		\acro{AFIMA }{Autoregressive Fractionally Integrated Moving Average} 
		\acro{VAR}{Vector Autoregression Model}
		\acro{EVI}{Enhance Vegetation Index}
		\acro{TMin and TMax}{Minimum and Maximum Temperature}
		\acro{ADF}{Augmented Dickey Fuller}
		\acro{OLS}{Optimal Lag Length Selection Criteria}
		\acro{AIC}{	Akaike Information Criterion}
		\acro{HQ}{Hannan-Quinn criterion}
		\acro{SC}{Schwarz Criterion}
		\acro{FPE}{Final Prediction Error criterion}
		\acro{FEVD}{Forecast Error Variance Decomposition}
	\end{acronym}

	%\renewcommand{\@makechapter}[]{%
		%	\vspace*{20 pt}%
		%	{\setlength{\parindent}{0pt} \raggedright \centering \bf \bfseries\normalfont{\thechapter \#1\par\nobreak\vspace{20 pt}}}}
	%\pagebreak
	\makeatother
	\titleformat{\chapter}{\bfseries\centering\normalfont\bf}{}{1em}{}
	\titleformat{\section}{\bfseries\normalfont\bf}{\thesection}{1em}{}
	
	\newpage
	\pagenumbering{arabic}
	\begin{flushleft}
		% Chapter 1

\chapter{CHAPTER ONE\\1.0 INTRODUCTION} % Main chapter title
\section{Introduction}
\label{Chapter1} % For referencing the chapter elsewhere, use \ref{Chapter1} 

%-------------------------------------------------------------------------------------

One would anticipate that the majority of emerging nations, which are still in the early stages of economic development and growth, would have a high forest cover and little deforestation. This, however, has not been the case. Ghana is a lower-middle-income nation that is still working toward middle-income classification. However, it has already begun to see a deforestation rate that is comparable to that of middle-income countries. The rapid population expansion, clearing of field for Galamsey operation,increased domestic need of wood for things like fuel, furniture, construction, and timber exports have all contributed to this trend, Bush fires in the 1980s, climate change, and lax law enforcement have all had an impact.

The purpose of this paper is to establish an understanding in time series analysis on remotely sensed data. Which will introduced us to the fundamentals of time series modeling, including decomposition, autocorrelation and modeling historical changes in Galamsey Operation in Ghana, the Cause,Dangers and it's Environmental impact.

Galamsey also known as "gather them and sell",\parencite{Mantey2017} is the term given by local Ghanaian for illegal small-scale gold mining in Ghana . The major cause of Galamsey is unemployment among the youth in Ghana \parencite{Gracia2018}. Young university graduates rarely find work and when they do it hardly sustains them. The result is that these youth go the extra mile to earn a living for themselves and their family.

Another factor is that lack of job security. On November 13, 2009 a collapse occurred in an illegal, privately owned mine in Dompoase, in the Ashanti Region of Ghana. At least 18 workers were killed, including 13 women, who worked as porters for the miners. Officials described the disaster as the worst mine collapse in Ghanaian history \parencite{BBCNews2009}.

Illegal mining causes damage to the land and water supply \parencite{Ansah2017} . In March 2017, the Minister of Lands and Natural Resources, Mr. John Peter Amewu, gave the Galamsey operators/illegal miners a three-week ultimatum to stop their activities or be prepared to face the law \parencite{Allotey2017} . The activities by Galamseyers have depleted Ghana's forest cover and they have caused water pollution, due to the crude and unregulated nature of the mining process \parencite{Gyekye}.

Under current Ghanaian constitution, it is illegal to operate as galamseyer.That is to dig on land granted to mining companies as concessions or licenses and any other land in search for gold. In some cases, Galamseyers are the first to discover and work extensive gold deposits before mining companies find out and take over. Galamseyers are the main indicator of the presence of gold in free metallic dust form or they process oxide or sulfide gold ore using liquid mercury.

Between 20,000 to 50,000, including thousands from China are believed to be engaged in Galamsey in Ghana.But according to the Information Minister 200,000 and nearly 3 million people, recently are now into Galamsey operation and rely on it for their livelihoods \parencite{goldgu2017}. Their operations are mostly in the southern part of Ghana where it is believe to have substantial reserves of gold deposits, usually within the area of large mining companies \parencite{Barenblitt2021} . As a group, they are economically disadvantaged. Galamsey settlements are usually poorer than neighboring agricultural villages. They have high rates of accidents and are exposed to mercury poisoning from their crude processing methods. Many women are among the workers, acting mostly as porters for the miners.

\section{Background of The Study}

As Galamsey is considered an illegal activity, they operations are hidden to the eyes of the authorities.So locating them is quite tricky ,but with satellite imagery ,it now possible to locate their operating and put an end to it. One of the features of Google Earth Engine is the ability to access years of satellite imagery without needing to download, organize, store and process this information. For instance, within the Satellite image

collection, now it possible to access imagery back to the 90's, allowing us to look at areas of interest on the map to visualize and quantify how much things has changed over time. With Earth Engine, Google maintains the data and offers it's computing power for processing.Users can now access hundreds of time series images and analyze changes across decades using GIS and R or other programming language to analyze these datasets.

\section{Problem Statement}
The Footprint of Galamsey is Spreading at a very faster rate, causing vegetation loss.Other factors accounting to vegetation loss may largely include climate change,urban and exurban development, bush fires. But not much works or research has been done to tell the extent to which Galamsey causes vegetation loss. This research attempts to segregate the variability climate is responsible for in vegetation loss so as to attribute the residual variability to Galamsey and other related activities such as bush-fires etc.

\section{Research Questions}
To address the challenge of the vegetation variability in this work, the following several statements were formed:

\begin{itemize}
	\item  Are there any changes in vegetation cause by Galamsey and Climate change in Ghana?
	\item Is there any relationship between vegetation loss and Climate change in Ghana?
\end{itemize}

\section{Research Objectives}
The purpose is to establish an understanding of time series analysis on remotely sensed data. We will be introduced to the fundamentals of time series modeling, including decomposition, auto-correlation, and modeling historical changes.Unfortunately, the causes of deforestation and forest degradation have not been adequately defined in the environmental literature. According to \paragraph{hosonuma2012assessment}, there are four causes of forest degradation: logging for wood, uncontrolled fires, livestock grazing in forests, and
five deforestation drivers, and fuel (wood/charcoal) (commercial agriculture, subsistence agriculture, mining, infrastructure and urban expansion). According to these drivers, this study reclassifies the drivers for an empirical examination into human conduct or activity and climatic change by;

\begin{itemize}
	\item Performing time series analysis on satellite derived vegetation indices
	
	\item Estimate the extent to which Galamsey causes vegetation loss in Ghana.
	
	\item Single out the variability climate is responsible for.
\end{itemize}

\section{Significance Of The Study}

There have been significant changes in vegetation cover in Ghana over the past 30 years, and these dynamics are related strongly to climatic factors such as temperature and other factors. In this study, we want to examine the effects of climatic change on Ghana's vegetation during these thirty years.

This study allows us to explore climatic differences and climate-related drivers. Additionally, it offers a chance to research how climatic variability affects the ecosystem and human health. By merging climatic and vegetation (EVI) data to understand the relationship between vegetation and climate change under tropical climate conditions, it closes research gaps in Ghana. This study explores historical and projected vegetation and climate data, by sector, impacts, key vulnerabilities and what adaptation measures can be taken. It also explores the overview for a general context of how climate change is affecting Ghana.

\section{Limitation Of The Study}

There have been significant changes in vegetation cover in Ghana over the past 30 years, and these dynamics are related strongly to climatic factors such as temperature and other factors. In this study, we want to examine the effects of climatic change on Ghana's vegetation during these thirty years.

This study allows us to explore climatic differences and climate-related drivers. Additionally, it offers a chance to research how climatic variability affects the ecosystem and human health. By merging climatic and vegetation data to understand the relationship between vegetation and climate change closes research gaps in Ghana. This study explores historical and projected vegetation and climatic data, by sector, impacts, key vulnerabilities and what adaptation measures can be taken. It also explores the overview for a general context of how climate change is affecting Ghana's Vegetation .%including chapters as standalone files
		% Chapter 2

\chapter{CHAPTER TWO\\2.0 LITERATURE REVIEW} % Main chapter title

\label{Chapter2} % For referencing the chapter elsewhere, use \ref{Chapter1} 

%----------------------------------------------------------------------------------------

% Define some commands to keep the formatting separated from the content 
%\newcommand{\keyword}[1]{\textbf{#1}}
%\newcommand{\tabhead}[1]{\textbf{#1}}
%\newcommand{\code}[1]{\texttt{#1}}
%\newcommand{\file}[1]{\texttt{\bfseries#1}}
%\newcommand{\option}[1]{\texttt{\itshape#1}}

%----------------------------------------------------------------------------------------
\section{Introduction}
According to studies, there is now significant change in vegetation on the earth than there was thirty years ago, and it is distributed differently.

More than half of the changes they found are attributed to the consequences of a warmer climate, with people only being responsible for
about a third. Perhaps surprisingly, they are unable to definitively link approximately 10\% of the changes to either the climate or us.\parencite{alex2013}

Several models and hypotheses have been established in the environmental literature to explain the relationship between human behaviour, and
environmental (forest) deforestation or depletion. Recent environmental and energy economics literature focuses on the energy consumption
choices made by businesses and people in developing countries Gertler etal. \parencite{gertler2016}. Africa's energy supply
is made mainly of fuel wood and charcoal to a degree of about 58\%. Specht et al. \parencite{specht2015} . Before other
demands for forest goods like furniture and paper, the need for fuel wood for cooking and heating is frequently identified as the main driver
of deforestation.

The causes of tropical forest decline are unclear, according to DeFries et al. \parencite{defries2010} . However, scientists were able to pinpoint the two primary causes of deforestation in the 21st century using information from satellite-based estimations in 41 different countries. The authors found a favorable association between forest loss and increases in urban population as well as agricultural exports using two methods of regression analysis. The same proof, however, was not discovered in the case of the increase in rural
population. This suggests that forest loss is unavoidable in regions with high levels of human activity.
Sedjo and Sohngen (1998) assessed various causes of climate-related forest degradation.  Change and its effects on society and the economy. They discovered a positive correlation between  forests in general, climate change, and timber harvesting. This suggests that earlier  Research results indicating serious implications have inflated the risk. They also assert that  Concern exists over how climate change will affect the ecological values of forests.  particularly if climate change occurs relatively gradually and its response is improved.
Using a single DGVM and meteorological data from 15 distinct climate models for low and high emissions, Salazar et al. (2007) investigated the impact of climate change on the Amazon forest. They discovered from the models that rising temperatures are sufficient to induce the loss of forest and the conversion of the Amazon forest to savannah, even with high rainfall. This is true despite the wide range of expected precipitation changes over the Amazon forest. However, given that several DGVMs produce varied results, there are issues with effectiveness, consistency, and dependability because just one DGVM was used in their study. According to a related study, Huntingford et al. (2008) predicted that the Amazonian rain forest would see some attrition in the 21st century as a result of climate change.They also noted that the expected variations in temperature and rainfall have an impact on how much attrition will occur. The scientists did stress that because of a lag in the reaction to climate change, the decrease in the forest will be worse than most forecasts.

Mean while DGVMs have been used in other empirical investigations to understand climate change and its effects on forests. The method replicates competition between various vegetation kinds and forecasts potential changes in wooded regions due to a warming climate. To understand the mechanisms underlying changes in vegetation types and cover, Reu et al. (2011) investigated the link between climatic change and plant physiological processes. They discovered that when the climate warms, the concentration of species declines in the tropics but grows in the mid-latitudes.

The effects of temperature and precipitation fluctuations on humidity in forests are significant. The efficiency with which plants use water can be impacted by increasing water losses via evaporation and evapotranspiration, according to Mortsch (2006). When a warm temperature persists for a prolonged amount of time—for example, over a drought develops, significant moisture stress occurs. Depending on the characteristics of the forest, such as the type of habitat for fauna and flora, soil depth, and soil type, this process results in a decline in the development and health of trees.




%---------------------------------------------------------------------------------------- 
		% Chapter 3

\chapter{CHAPTER THREE\\3.0 METHODOLOGY} % Main chapter title

\label{Chapter3} % For referencing the chapter elsewhere, use \ref{Chapter1} 

%-----------------------------------------------------------------------------------

\section{Introduction}
The techniques used to study our model are outlined in this chapter, together with a comprehensive explanation of the mathematical tools and constructs, theorems, lemmas, and their justifications.  The planning and execution processes of our technique make use of the R programming language. During the planning process, we considered where to get our data from and the procedures needed to build a time series model from the satellite data. We categorize our research as taking a quantitative approach. This research is specifically a causal-comparative experimental study with the goal of identifying the variability climate is responsible for in vegetation loss in Ghana.
 \section{Study Area}
 The Republic of Ghana, a nation in West Africa, will serve as the location for the experimental plots for this study. It shares borders with the Ivory Coast in the west, Burkina Faso in the north, and Togo in the east. It borders the Gulf of Guinea and the Atlantic Ocean to the south. Ghana's total size is 238,535 km2 (92,099 sq mi), and it is made up of a variety of biomes, from tropical rainforests to coastal savannas. Ghana, which has a population of over 31 million, is the second-most populous nation in West Africa, behind Nigeria.Accra, the nation's capital and largest city, as well as Kumasi, Tamale, and Sekondi-Takoradi, are other important cities

\section{DATA }
Data gathering. The majority of the quantitative data came from the Ghana Meteorological Agency and the Health Service. Rainfall, the highest temperature, and relative humidity are some of the meteorological data that were measured from 2010 to 2015 on a mean monthly basis. Additionally, based on monthly incidences for the Kumasi Metropolitan Area from 2000 to 2022, the data for EVI were obtained.

\subsection{Data Description and Inspection}

Data from a time series is a set of observations made in a particular order over a period of time. There is a chance for correlation between observations because time series data points are gathered at close intervals. To help machine learning classifiers work with time series data, we provide several new tools. We first contend that local features or patterns in time series can be found and combined to address challenges involving time-series categorization. Then, a method to discover patterns that are helpful for classification is suggested. We combine these patterns to create computable categorization rules. In order to mask low-quality pixels, we will first collect data from Google Earth Engine in order to choose  EVI values and Climate Change data.

Instead of analyzing the imagery directly, we will summarize the mean  Climate values. This will shorten the analysis time while still providing an attractive and useful map. We will apply a smoothing strategy using an ARIMA function to fix the situation where some cells may not have  EVI for a particular month. Once NA values have been eliminated, the time series will be divided to eliminate seasonality before the normalized data is fitted using a linear model. We will go to classify our data on the map and map it after we have extracted the linear trend.

Here, we made sure that we checked our data set for missing values and discovered there were none. The dataset’s number of columns is shown in the table below,
which also details the counts and datatype of various features in our dataset. No values are missing when there is a count of 133990. Anything below that denotes
the existence of unreported values.


 \subsection{Time Series Forecasting Using Stochastic Models}
 The selection of a proper model is extremely important as it reflects the underlying structure of the series and this fitted model in turn is
 used for future forecasting. A time series model is said to be linear or non-linear depending on whether the current value of the series is a
 linear or non-linear function of past observations.
 
 In general models for time series data can have many forms and represent different stochastic processes. There are two widely used linear time
 series models in literature.
 
 \emph{Autoregressive (AR)} and \emph{Moving Average (MA)} models, combining these two, the Autoregressive Moving Average (ARMA) and
 \emph{Autoregressive Integrated Moving Average (ARIMA)} models have been proposed in many literature. The \emph{Autoregressive Fractionally
 	Integrated Moving Average (ARFIMA)} model generalizes ARMA and ARIMA models. For seasonal time series forecasting, a variation of ARIMA. The
 \emph{Seasonal Autoregressive Integrated Moving Average (SARIMA)}  model is used.
 
 ARIMA model and its different variations are based on the famous Box-Jenkins principle Hipel and McLeod \parencite{hipel1994} and so these are also broadly known as the Box-Jenkins models.
\textbf{Vector Autoregression Model}\\
Model of vector autoregression. In econometrics, the Sims' vector autoregression (VAR) model has been widely employed to analyze multivariate time series. It provides better forecasting abilities than a univariate time series model and is a logical extension of the univariate autoregressive model to dynamic multivariate time series. Additionally, it identifies how each endogenous variable responds over time to a shock in both its own value and in every other variable. This method also enables the researcher to follow the data. The VAR model of order $p$ Sims \parencite{sims1980macroeconomics} proposed has the following basic form:
\begin{equation}
	y_{t} = A_{1}y_{t-1} + A_{2}y_{t-2} +\cdot+A_{p}y_{t-p}+ CD_{t} + \mu 
\end{equation}
Where $ y_{t} = \left( y_{1t},y_{2t},...,y_{kt}\right)'$ is a vector of K observable endogenous variables. For the purposes of this study, yt = $(EVI_{t}, Temperature_{t}, Rain_{}, Precipitation_{t},Drought_{t},Evaporation_{t})'$ where EVI denotes the value of vegetation conditions  each month, Temperature for both TMax and TMin, Rain denotes the amount of precipitation (mm), and Drought denotes the relative drought (\%). All deterministic variables, including constants, linear trends, and seasonal dummy variables, are contained in $CD_{t}$. An unobservable zero-mean white noise process in K dimensions, $\mu_{t}$ has a positive definite covariance matrix $E(\mu_{t},\mu_{t}^{'}) = \sum_{\mu}$. We  apply different limits on the parameter matrices $A_{i}$ and C, which are of an appropriate dimension.
Generalized least squares is used to estimate the model's parameters.

\textbf{Augmented Dickey Fuller Test.} For stationarity, the Augmented Dickey Fuller \textbf{(ADF)} test is a unit root test. The alternative hypothesis varies slightly depending on the equation used, whereas the null hypothesis for the ADF test is that there is a unit root. The time series is steady, which is the most straightforward alternate theory. In time series analysis, unit roots can lead to unexpected outcomes.
\begin{align}        \Delta y_t &=  \color{red}\gamma y_{t-1} + \underbrace{\sum_{i=2}^{p} \beta_i \Delta y_{t-i+1}}_\text{control for serial correlation} + \epsilon_t  \\ \rightarrow (\tau1&)\quad H_0 : \color{red} {\gamma = 0} \\\\       \Delta y_t &=  \color{red}{\gamma} y_{t-1}  + \underbrace{\color{blue}{a_0}}_\text{constant} + \sum_{i=2}^{p} \beta_i \Delta y_{t-i+1} + \epsilon_t   \\ \rightarrow (\phi1&)\quad H_0 : \color{red} {\gamma = 0} \quad\&\quad\color{blue}{a_0 = 0}  \\\ \rightarrow (\tau2&)\quad H_0 : \color{red} {\gamma = 0}  \\\       \Delta y_t &=  \color{red}{\gamma} y_{t-1}  + \color{blue}{a_0}  + \underbrace{\color{green}{a_2} t}_{trend} + \sum_{i=2}^{p} \beta_i \Delta y_{t-i+1}  + \epsilon_t \\ \rightarrow (\phi2&)\quad H_0 : \color{red} {\gamma = 0} \quad\&\quad \color{blue}{a_0 = 0} \quad\&\quad \color{green}{a_2 = 0} \\ \rightarrow (\phi3&)\quad H_0 : \color{red} {\gamma = 0} \quad\&\quad\color{blue}{a_0 = 0}  \\ \rightarrow (\tau3&)\quad H_0 : \color{red} {\gamma = 0}    
 \end{align}

\textbf{Optimal Lag Length Selection Criteria (OLS)}  is used to estimate each of the system's individual equations. One of the following Information Criteria is minimized to determine the best lag order that is choosing optimal lag  to reduce residual correlation
\begin{center}
		\begin{tabular}{ll}
			\multicolumn{1}{c}{}Criteria &  Formular \\ \midrule
			Akaike Information Criterion,AIC (n)  & = log det $ \displaystyle\left(\sum _{\mu}(n)\right)+ \frac{2}{T}nK^{2}$\\
			Hannan-Quinn criterion,HQ(n)    & = log det  $ \displaystyle\left(\sum _{\mu}(n)\right)+ \frac{2logT}{T}nK^{2}$\\
			Schwarz Criterion,SC(n)    & = log det  $ \displaystyle\left(\sum _{\mu}(n)\right)+ \frac{logT}{T}nK^{2}$\\
			Final Prediction Error criterion,FPE(n)   & =          $\displaystyle \left(\frac{T + n^{*}}{T - n^{*}}\right)$det $\displaystyle \left(\sum_{\mu}(n)\right)$\\ \bottomrule
		\end{tabular}
\end{center}
where $ \displaystyle \sum_{\mu} (n)$ is the estimated by $T^{-1}\sum_{t=1}^{T}$
\section{Structural Analysis}
There are frequently many coefficients to comprehend, despite the fact that VAR coefficients describe the projected impact of a variable. Examining the model's residuals, which represent unforeseen contemporaneous events, is typically more prevalent. The next subsections explain some of the typical methods used for structural analysis of VAR models in a manner that is comparatively nontechnical.
\subsection{Causality Analysis} Both the Granger-causality and instantaneous causality were investigated. For both tests, the vector of endogenous variables is divided into two subvectors,$y_{1t}$ and $y_{2t}$ with dimensions $K_{1}$ and $K_{2}$,ctively, so that $K = K_{1} + K_{2}$ The subvector $y_{1t}$ said to be Granger-causal for
$y_{2t}$ if the past of $y_{1t}$ significantly helps predicting the future of $y_{2t}$ a the past of $y_{1t}$ one \parencite{granger1969investigating}. For testing this property,
a model of the form
\begin{equation}
	\displaystyle
	\left[ \begin{array}{c} y_{1t} \\ y_{2t}  \end{array} \right] \mbox{=} \sum_{i=1}^{p}\left[\begin{array}{cc}
		\alpha_{11,i} & \alpha_{12,i} \\
		\alpha_{21,i} & \alpha_{22,i} 
	\end{array}\right]\left[\begin{array}{c}
		y_{1,t-i}\\
		y_{2,t-i}
	\end{array}\right] + CD_{t} + \left[ \begin{array}{c}
		\mu_{1t}\\
		\mu_{2t}
	\end{array}\right]
\end{equation}
Where, $y_{1t}$ is not considered as  Granger-causal for $y_{2t}$ if and only if $ \alpha_{21,i} = 0,$ i = 1,2,...p.
Therefore, this null hypothesis is tested that at least one of the $\alpha_{21,i}$, has a positive value. The constraints are tested using an F-test statistic with a distribution of $F(pK_{1}K_{2}, KT - n^{*})$. Here, $n^{*}$ is the entire number of system parameters, including those for the deterministic term \parencite{lutkepohl2005new}. To test Granger-cause from $y_{2t}$ to $y_{1t}$, the roles of $y_{1t}$ and $y_{2t}$ can be switched.In other words, Granger causality responds to the question of whether previous values of the variable x may help anticipate future values of the variable y.

Instantaneous causality is defined as having a nonzero correlation between $\mu_{1t}$ and $\mu_{2t}$. Consequently, the null hypothesis is tested against the alternative 
\begin{equation}
	H_{0}: E\left(\mu_{1t},\mu_{2t}^{'} \right)
\end{equation}
is checked for instantaneous causation versus the alternative of nonzero covariance between the two error vectors. This hypothesis is investigated using a Wald test statistic.
\textbf{Analysis of impulse responses}: Exogenous and deterministic variables are viewed as fixed in impulse response analysis and can thus be removed from the system. Now, $y_{t}$ stands for the adjusted endogenous variables. The process $y_{t}$ has a Wold moving average (MA) representation if it is stationary \textbf{(I(0))}, which is represented by 
\begin{equation}
	y_{t} = \varPhi_{0}\mu_{t} + \varPhi_{1}\mu_{t-1} + \varPhi_{2}\mu_{t-2} + \dots
\end{equation}
Where $\varPhi_{0} =I_{K}$ and the $\varPhi_{s}$ can be computed recursively as 
\begin{equation}
	\varPhi_{s} = \sum_{j=i}^{s}\varPhi_{s-j}A_{j}, \quad s = 1,2,\dots,
\end{equation}
With $\varPhi = I_{K}$ and $A_{j} = 0 $ for $j>p$. The coefficients of this
representation may be interpreted as reflecting the responses
to impulses hitting the system. The \emph{(i,j)th} element of the matrices $\varPhi_{s},$ regarded as a function of \emph{s}, trace out the expected
response of $y_{i,t+s}$ to a unit change in $y_{it}$ holding constant all
past values of $y_{t}$
\section{ Forecast Error Variance Decomposition}
Denoting the $(i,j)th$ element of the orthogonalized impulse response
coefcient matrix $\theta_{n}$ by $\theta_{ij,n}$, the variance of the forecast error $\left(y_{k,T+h}-y_{k,T+h|T}\right)$ is 
\begin{equation}
	\begin{split}
		\displaystyle
		\sigma^{2}_{k}(h) & = \sum^{h-1}_{n=0}\left(\theta^{2}_{k1,n} + \cdot+ \theta^{2}_{kK,n}\right) \\
		& =\sum^{K}_{j=1}\left(\theta^{2}_{kj,0} + \cdot+ \theta^{2}_{kj,h=1}\right)
	\end{split}
\end{equation}
The term $\displaystyle \left(\theta^{2}_{kj,0} + \cdot+ \theta^{2}_{kj,h=1}\right) $ is interpreted as the contribution
of variable \emph{j} to the \emph{h}-step forecast error variance of variable
\emph{k}.  Dividing the above terms by $\displaystyle \sigma^{2}_{k}(h)$ gives the percentage \emph{j} to the \emph{h}-step forecast error variance of variable
\emph{k}, 
\begin{equation}
	\omega_{kj}(h) =\frac{\left(\theta^{2}_{kj,0} + \cdot+ \theta^{2}_{kj,h=1}\right)}{\sigma^{2}_{k}(h)}
\end{equation}
\section{Forecast Performance Measures}

While applying a particular model to some real or simulated time series, first the raw data is divided into two parts (\textbf{Training Set and
	Test Set}). The observations in the training set are used for constructing the desired model. Often a small sub-part of the training
set is kept for validation purpose and is known as the \textbf{Validation Set}. Sometimes a preprocessing is done by normalizing the data or taking logarithmic or other transforms. One such famous technique is the Box-Cox Transformation {[}23{]}. Once a model is constructed, it is used for generating forecasts. The test set observations are kept for verifying how accurate the fitted model performed in forecasting these values. If necessary, an inverse transformation is applied on the forecast values to convert them in original scale. In order to judge the forecasting accuracy of a particular model or for evaluating and comparing different models, their relative performance on the test dataset is considered.

Due to the fundamental importance of time series forecasting in many practical situations, proper care should be taken while selecting a particular model. For this reason, various performance measures are proposed in literature to estimate forecast accuracy and to compare different models. These are also known as performance metrics {[}24{]}. Each of these measures is a function of the actual and forecast values of the time series.

% this line fixes the vertical padding of text inside the cells
\renewcommand{\arraystretch}{1.4}
%\begin{sidewaystable}
%	\caption{Description of Various Forecast Performance Measures}
%	\label{tab:my-table}
%	\resizebox{\columnwidth}{2.5cm}
%	\begin{tabularx}{\textwidth}{|L{2cm}|X|L{2cm}|L{2cm}|X|}{lllll}
%		\hline
%		& \multicolumn{4}{l}{\cellcolor[HTML]{C0C0C0}}{Error Matrix} \\ \midrule    
%		
%		& \multicolumn{1}{l}{MAE}
%		& \multicolumn{1}{l}{RMSE}
%		& \multicolumn{1}{l}{MSE}
%		& \multicolumn{1}{l}{MAPE} \\
%		\midrule
%		\multicolumn{13}{l}{}                                \\
%		\\\multicolumn{13}{l}{}   \\ 
%		
%		\multicolumn{13}{c}{Where: In each of the forthcoming definitions, $y_{t }$ is the actual value,$f_{t}$ is the forecast value, $e_{t} = y_{t} - f_{t}$ is the forecast error and n is the size of the test set. Also, $\displaystyle \bar{y} = \frac{1}{n}\sum_{t=1}^{n}y_{t}$ is the test mean and $\displaystyle \sigma^{2} = \frac{1}{n-1}\sum_{t=1}^{n}(y_{t}-\bar{y})^{2}$is the test variance.}                                \\\hline\hline
%		Formulas 
%		& $\displaystyle MAE = \frac{1}{n}\sum_{t=1}^{n}|e_{t}|$ \\
%		& $\displaystyle RMSE = \sqrt{MSE} = \sqrt {\frac{1}{n}\sum_{t=1}^{n}e^{2}_{t}}$ \\
%		& $\displaystyle MSE = \frac{1}{n}\sum_{t=1}^{n}e^{2}_{t}$\\
%		& $\displaystyle MAPE= \frac{1}{n}\sum^{n}\left(\frac{e_{t}}{y_{t}}\right)$ \\ \bottomrule
%		
%		\multicolumn{13}{l}{}                                \\
%	\end{tabularx}%
%	
%	
%\end{sidewaystable}
\newpage
\subsection{Description of Various Forecast Performance Measures}
\textbf{MAPE}\\
This measure represents the percentage of average absolute error occurred. It is independent of the scale of measurement, but affected by data transformation.It does not show the direction of error. MAPE does not penalize extreme deviations. In this measure, opposite signed errors do not offset each other.
\\
\textbf{MSE}\\
It is a measure of average squared deviation of forecast values. As here the opposite signed errors do not offset one another, MSE gives an overall idea of the error occurred during forecasting. It penalizes extreme errors occurred while forecasting. MSE emphasizes the fact that the total forecast error is in fact much   affected by large individual errors, i.e. large errors are much expensive than small errors. MSE does not provide any idea about the direction of overall error. MSE is sensitive to the change of scale and data transformations. Although MSE is a good measure of overall forecast error, but it is   not as intuitive and easily interpretable as the other measures discussed before. 

\textbf{RMSE}\\
RMSE is nothing but the square root of calculated MSE. All the properties of MSE hold for RMSE as well.

\textbf{MAD}\\
measures the average absolute deviation of forecast values from   original ones.It shows the magnitude of overall error, occurred due to forecasting.In MAE, the effects of positive and negative errors do not cancel out.Unlike MFE, MAE does not provide any idea about the direction of  errors. For a good forecast, the obtained MAE should be as small as possible. Like MFE, MAE also depends on the scale of measurement and data   transformations.Extreme forecast errors are not panelized by MAE.\\
%\begin{landscape}
	\begin{table}[]
		\begin{tabularx}{\textwidth}{@{}llll@{}}
			\toprule \hline
			MAE	&  RMSE   &  MSE   &  MAPE  \\ \midrule
			\hline
			$\displaystyle  \frac{1}{n}\sum_{t=1}^{n}|e_{t}|$	& $\displaystyle  \sqrt{MSE} = \sqrt {\frac{1}{n}\sum_{t=1}^{n}e^{2}_{t}}$    & $\displaystyle  \frac{1}{n}\sum_{t=1}^{n}e^{2}_{t}$    & $\displaystyle  \frac{1}{n}\sum^{n}\left(\frac{e_{t}}{y_{t}}\right)$   \\ \bottomrule
			
		\end{tabularx}
	\end{table}
%\end{landscape}
Where: In each of the forthcoming definitions, $y_{t }$ is the actual value,$f_{t}$ is the forecast value, $e_{t} = y_{t} - f_{t}$ is the forecast error and n is the size of the test set. Also, $\displaystyle \bar{y} = \frac{1}{n}\sum_{t=1}^{n}y_{t}$ is the test mean and $\displaystyle \sigma^{2} = \frac{1}{n-1}\sum_{t=1}^{n}(y_{t}-\bar{y})^{2}$is the test variance.


		% Chapter 3

\chapter{CHAPTER FOUR\\4.0 ANALYSIS, RESULTS \& DISCUSSION} % Main chapter title

\label{Chapter4} % For referencing the chapter elsewhere, use \ref{Chapter1} 

%----------------------------------------------------------------------------------------

% Define some commands to keep the formatting separated from the content 
%\newcommand{\keyword}[1]{\textbf{#1}}
%\newcommand{\tabhead}[1]{\textbf{#1}}
%\newcommand{\code}[1]{\texttt{#1}}
%\newcommand{\file}[1]{\texttt{\bfseries#1}}
%\newcommand{\option}[1]{\texttt{\itshape#1}}

%------------------------------------------------------------------------------------------------


\subsection{Variable \& Parameter Definition}
Find in the tables (\textbf{4.2}) and (\textbf{4.3}) for the definition of variables and parameters respectively as used in this study. 
\begin{table}
	\label{Variable}
	\caption{Variable Definition.}
	\centering
\begin{tabular}{ll}
	\hline\noalign{\smallskip}
	Variable & Definition \\ 
	\noalign{\smallskip}\hline\noalign{\smallskip}
	S & Susceptible individuals \\  
	E & Exposed individuals \\ 
	I & Infected individuals \\ 
	R & Recovered and Immunized individuals \\ 
	\hline\noalign{\smallskip}
\end{tabular}
\end{table}
\begin{table}
\label{tab:P}
\caption{Parameter Definition}
\centering
\begin{tabular}{ll}
	\hline\noalign{\smallskip}
	Parameters & Definitions\\ 
	\noalign{\smallskip}\hline\noalign{\smallskip}
	$\alpha$ & New births and Immigration rates\\ 
	$\mu$ & Mortality rate \\ 
	$\mu_{E}$ & $(\mu + k_{E})$, where $k_{E}$ is the rate at which exposed individuals die\\
	$\lambda$ & Proportion of children vaccinated successfully at birth \\ 
	$\beta$ & Average number of adequate contacts of a person per unit time \\ 
	$\sigma$ & The rate at which exposed individuals get infectious \\ 
	$\gamma$ & The rate at which infected individuals get recovered \\ 
	$ \delta_{E} $ & Number of natural immunes per thousands\\
	$\tau$ & $(\mu + k_{I})$, where $k_{I}$ is the rate at which infected individuals die\\
	\hline\noalign{\smallskip}
\end{tabular}
\end{table}
%-----------------------------------------------------------------------------------
\subsection{Model Equations}
The following is the model equation gleaned from the schematic diagram above - Figure (\textbf{4.1})  for the formulation of a SEIRS model;
\begin{equation}
\label{eq4.1}
\begin{aligned}
S^{'} &= \alpha N + \epsilon R - \left(\mu + \beta I/N + \lambda\right)S \\
E^{'} &= \beta S I/N -\left(\mu_{E} + \sigma + \delta_{E}\right)E \\
I^{'} &= \sigma E -(\tau + \gamma)I\\
R^{'} &= \gamma I - (\mu + \epsilon)R + \lambda S + \delta_{E}E
\end{aligned}
\end{equation}
And we find the dynamics \ref{eq4.1} of the total population by adding the equations above to get;
\begin{equation}
\begin{split}
\begin{aligned}
S^{'} + E^{'} + I^{'} + R^{'} &= \alpha N + \epsilon R - \left(\mu + \beta I/N + \lambda\right)S + \beta S I/N\\
&\quad-\left(\mu_{E} + \sigma + \delta_{E}\right)E + \sigma E -(\tau + \gamma)I\\
&\quad + \gamma I - (\mu + \epsilon)R + \lambda S + \delta_{E}E
%%&\quad 
\end{aligned}
\end{split}
\end{equation}

%--------------------------------------------------------------------------------------------------


%-----------------------------------------------------------------------------------
\begin{equation}
\label{xtics}
\begin{aligned}
F(\lambda) &= -\frac{\beta {h_1} {h_3} {h_4} {k_I} \sigma}{{h_3} {h_4} {k_I}-\beta {h_3} {h_4}}-\frac{\lambda \beta {h_1} {h_4} {k_I} \sigma}{{h_3} {h_4} {k_I}-\beta {h_3} {h_4}}-\frac{\lambda \beta {h_1} {h_3} {k_I} \sigma}{{h_3} {h_4} {k_I}-\beta {h_3} {h_4}}-\frac{{{\lambda}^{2}} \beta {h_1} {k_I} \sigma}{{h_3} {h_4} {k_I}-\beta {h_3} {h_4}}\\ \quad&+\frac{{{\beta}^{2}}\, {h_1} {h_3} {h_4} \sigma}{{h_3} {h_4} {k_I}-\beta {h_3} {h_4}}+\frac{\lambda\, {{\beta}^{2}}\, {h_1} {h_4} \sigma}{{h_3} {h_4} {k_I}-\beta {h_3} {h_4}}+\frac{\lambda\, {{\beta}^{2}}\, {h_1} {h_3} \sigma}{{h_3} {h_4} {k_I}-\beta {h_3} {h_4}}+\frac{{{\lambda}^{2}}\, {{\beta}^{2}}\, {h_1} \sigma}{{h_3} {h_4} {k_I}-\beta {h_3} {h_4}}\\ \quad &+\frac{{h_2} {{{h_3}}^{2}} {{{h_4}}^{2}} {k_I}}{{h_3} {h_4} {k_I}-\beta {h_3} {h_4}}+\frac{\lambda\, {h_2} {h_3} {{{h_4}}^{2}} {k_I}}{{h_3} {h_4} {k_I}-\beta {h_3} {h_4}}+\frac{\lambda\, {h_2} {{{h_3}}^{2}} {h_4} {k_I}}{{h_3} {h_4} {k_I}-\beta {h_3} {h_4}}+\frac{{{\lambda}^{2}}\, {h_2} {h_3} {h_4} {k_I}}{{h_3} {h_4} {k_I}-\beta {h_3} {h_4}}\\ \quad&-\frac{\beta {h_2} {{{h_3}}^{2}} {{{h_4}}^{2}}}{{h_3} {h_4} {k_I}-\beta {h_3} {h_4}}-\frac{\lambda \beta {h_2} {h_3} {{{h_4}}^{2}}}{{h_3} {h_4} {k_I}-\beta {h_3} {h_4}}-\frac{\lambda \beta {h_2} {{{h_3}}^{2}} {h_4}}{{h_3} {h_4} {k_I}-\beta {h_3} {h_4}}-\frac{{{\lambda}^{2}} \beta {h_2} {h_3} {h_4}}{{h_3} {h_4} {k_I}-\beta {h_3} {h_4}}\\ \quad&-\lambda\, {h_2} {h_4}-{{\lambda}^{2}}\, {h_4}-\lambda\, {h_2} {h_3}-{{\lambda}^{2}}\, {h_3}-{{\lambda}^{2}}\, {h_2}-{{\lambda}^{3}} 
\end{aligned}
\end{equation}

\subsubsection{Global Stability Analysis of the Endemic Equilibrium}
A general form of Lyapunov functions coined from the first integral of the Lokta-Volterra system which is often used in the literature of mathematical biology is used to prove the global stability of the EE. This function takes the form $$ \mathscr{L} = \Sigma_{i=1}^{n}c_{i}\left(x_{i} - x_{i}^{*} - x_{i}^{*}\ln\dfrac{x_{i}}{x^{*}_{i}}\right) $$ where $ x $ are the variables and $ c_{i} $ are carefully selected constants. This criterion has been used many times in establishing the stability or otherwise of many disease models and also present in \parencite{shuai2013global}.
\begin{theorem}
	The EE is globally stable.
\end{theorem}
\textit{Proof}.
Let 
\begin{equation}
\begin{aligned}
\mathscr{L}_{1} &= s - s^{*} - s^{*}\ln\frac{s}{s^{*}}\\
\mathscr{L}_{1}' &= -\left(\dfrac{s^{*} - s}{s}\right) s'\\
&\leq -\left(\dfrac{s^{*} - s}{s}\right)(h_{1}  -h_{2}s - (\beta - k_{I})si), \\
&\text{ and the equilibrium relation for $ h_{1} = h_{2}s^{*} + (\beta - k_{I})s^{*}i^{*} $ }\\
&\leq -\left(\dfrac{s^{*} - s}{s}\right)(h_{2}(s^{*} - s) + (\beta - k_{I})(s^{*}i^{*} - si) )\\
\mathscr{L}_{1}'&\leq 0, \mbox{ irrespective of the values assumed by $ s^{*} $, $ s $ $ i^{*} $ and $ i $ in the region $ \Re_{+} $} 
\end{aligned}
\end{equation}
Again, let 
\begin{equation}
\begin{aligned}
\mathscr{L}_{2} &= e - e^{*} - e^{*}\ln\frac{e}{e^{*}}\\
\mathscr{L}_{2}' &= -\left(\dfrac{e^{*} - e}{e}\right) e'\\
\mathscr{L}_{2}' &\leq -\left(\dfrac{e^{*} - e}{e}\right)(\beta si - h_{3}e)\\
&\leq -(e^{*} - e)\left(\dfrac{\beta si}{e} - h_{3}\right)\\
&\leq -(e^{*} - e)\left(\dfrac{\beta s^{*}i^{*}}{e^{*}} - h_{3}\right),\\
&\text{ then from the system under study, (\ref{4.12}), $ \dfrac{i^{*}}{e^{*}} = \dfrac{\sigma}{h_{4}} $ and $ s^{*} =  \dfrac{h_{3}h_{4}}{\beta\sigma} $ }\\
&\leq -(e^{*} - e)\left(\dfrac{\beta h_{3}h_{4}\sigma}{\beta h_{4}\sigma} - h_{3}\right) = -(e^{*} - e)\times 0 = 0\\
\mathscr{L}_{2}' &\leq 0 
\end{aligned}
\end{equation}
Lastly, let us set 
\begin{equation}
\begin{aligned}
\mathscr{L}_{3} &= i - i^{*} - i^{*}\ln\frac{i}{i^{*}}\\
\mathscr{L}_{3}' &= -\left(\dfrac{i^{*} - i}{i}\right) i'\\
\mathscr{L}_{3}' &\leq -\left(\dfrac{i^{*} - i}{i}\right) (\sigma e - h_{4}i )\\
&\leq -\left(i^{*} - i\right) \left(\sigma\dfrac{e}{i} - h_{4} \right)\\
&\leq -\left(i^{*} - i\right) \left(\sigma\dfrac{e^{*}}{i^{*}} - h_{4} \right), \text{ $ \dfrac{e^{*}}{i^{*}} = \dfrac{h_{4}}{\sigma}$ } \\
&\leq -\left(i^{*} - i\right) \left(\sigma\dfrac{h_{4}}{\sigma} - h_{4} \right) = 0\\
\mathscr{L}_{3}'&\leq 0
\end{aligned}
\end{equation}
Therefore $ \mathscr{L} $ defined as $ \mathscr{L} = \mathscr{L}_{1} + \mathscr{L}_{2}  + \mathscr{L}_{3} $ is a Lyapunov function for the system 10. Arbitrary constants $ c_{i} $ can be chosen from $ \Re_{+} $ and any linear combination of $ \mathscr{L} $ would be a Lyapunov function for the system. Hence the proof.

%------------------------------------------------------------------------------------------------

\begin{figure}
	\centering
	\includegraphics[width=0.7\linewidth]{images/Logo2}
	\caption{UENR Logo}
	\label{fig:logo2}
\end{figure}


%------------------------------------------------------------------------------------------------
 
		 \chapter{CHAPTER FIVE\\5.0 CONCLUSION \& RECOMMENDATIONS}
\label{Chapter5}
%--------------------------------------------------------------------------------------------------
\section{Introduction}
This chapter contains the summary of our findings and the recommendations from our findings. These recommendations are necessary information for the Vegetation Changes in Ghana and also for Mathematicians in the study of Time Series systems.

%--------------------------------------------------------------------------------------------------
\section{Conclusion}
This paper develops and estimates a Victor Autoregression (VAR) model of the monthly Vegetation condition and some
important climatic variables including precipitation, maximum temperature, and relative drought in the southern part of Ghana. The model is used to investigate the dynamic link  between vegetation and climatic variability. The model is also used to simulate the responses of EVI to innovations in climatic variability.
Results of the Granger causality tests lead to a conclusion that EVI is influenced by only three climatic variables for that particular study area(Cell number 196). The impulse response analyses indicate that the highest positive effect of  temperature, drought, and precipitation on EVI is observed in the ninth, third, and tenth months, respectively. The decomposition of forecast variance indicates varying degree of EVI dependence on the climatic variables, with as high as 12.65\% of the variability in the trend of EVI being explained by past innovations in  temperature alone. Results obtained from this study are useful for policy-makers as this will help come up with policies knowing the effects of climatic variability on EVI incidence in the Ghana.

\section{RECOMMENDATIONS}
Time series forecasting is a fast growing area of research and as such provides many scope for future works. One of them is the Combining Approach, i.e. to combine a number of different and dissimilar methods to improve forecast accuracy. A lot of works have been done towards this direction and various combining methods have been proposed in literature. Together with other analysis in time series forecasting, we have thought to find an efficient combining model, in future if possible. With the aim of further studies in time series modeling and forecasting.Therefore time series models, the rule of thumb is that one should have at least fifty (50) to sixty (60) data points but preferably more than hundred (100) observations \parencite{box1975intervention}. It is therefore suggested that future studies in this area of interest should consider more than hundred data points.. Therefore, it is recommended that future research in this area of interest take into account more than a hundred data points.Using  satellite data to help inform reclamation projects. Knowing the location and extent of degraded forests can help land managers better project the labor and expense to reclaim an area (by planting tree seedlings or adding plants that could detoxify the area, for instance)
	\end{flushleft} 	

	\printbibliography % displays the references cited.
	
	\appendix
	
	\addcontentsline{toc}{chapter}{APPENDICES}
	\chapter{Appendix Chapter 1}
	\section{Data Extraction Using R code From Google Earth Engine}
			\begin{shaded}
				\begin{verbatim}
				library(tidyverse)  # for data wrangling and visualization
				library('sf')
				library(tibble)
				library(lubridate)						
				library(rgee)			
				library(reticulate)
				ee_install()
				ee_check()
				
				ee_Initialize("kalong",drive = TRUE) # initialize GEE,
				#this will have you log in to Google Drive
				\end{verbatim}
			\end{shaded}
				
			Load shape file
			\begin{shaded}
				\begin{verbatim}
					{
						aoi <- read_sf('Ghana shp file/ROI/new_roi.shp')
						aoi <- st_transform(aoi, st_crs(4326))
						aoi.ee <- st_bbox(aoi) %>%
						st_as_sfc() %>%
						sf_as_ee() #Converts it to an Earth Engine Obj
					}			
					
					Date <- Sys.Date()
				\end{verbatim}
			\end{shaded}
			Map each image from 2000 to extract the monthly Climatic Data from the
			 Terraclimate dataset and rename the bands of the image
			
			\begin{shaded}
				\begin{verbatim}
					{
						Precipitation  <- ee$ImageCollection("UCSB-CHG/CHIRPS/DAILY") %>%
						ee$ImageCollection$filterDate("2000-01-01", rdate_to_eedate(Date)) %>%
						ee$ImageCollection$map(function(x) x$select("precipitation")) %>% 
						ee$ImageCollection$toBands() # from imagecollection to image
						
					}
					{
						MinimumTemperature <- ee$ImageCollection("IDAHO_EPSCOR/TERRACLIMATE") %>%
						ee$ImageCollection$filterDate("2000-01-01", rdate_to_eedate(Date)) %>%
						ee$ImageCollection$map(function(x) x$select("tmmn")) %>% 
						ee$ImageCollection$toBands()}
					{
						MaximumTemperature <- ee$ImageCollection("IDAHO_EPSCOR/TERRACLIMATE") %>%
						ee$ImageCollection$filterDate("2000-01-01", rdate_to_eedate(Date)) %>%
						ee$ImageCollection$map(function(x) x$select("tmmx")) %>% 
						ee$ImageCollection$toBands()}
					{
						Evapotranspiration <- ee$ImageCollection("NASA/FLDAS/NOAH01/C/GL/M/V001") %>%
						ee$ImageCollection$filterDate("2000-01-01", rdate_to_eedate(Date)) %>%
						ee$ImageCollection$map(function(x) x$select("Evap_tavg")) %>% 
						ee$ImageCollection$toBands()}
					{
						Humidity <- ee$ImageCollection("NASA/FLDAS/NOAH01/C/GL/M/V001") %>%
						ee$ImageCollection$filterDate("2000-01-01", rdate_to_eedate(Date)) %>%
						ee$ImageCollection$map(function(x) x$select("Qair_f_tavg")) %>% 
						ee$ImageCollection$toBands()}
					{
						Drought <- ee$ImageCollection("IDAHO_EPSCOR/TERRACLIMATE") %>%
						ee$ImageCollection$filterDate("2000-01-01", rdate_to_eedate(Date)) %>%
						ee$ImageCollection$map(function(x) x$select("pdsi")) %>% 
						ee$ImageCollection$toBands()}
				\end{verbatim}
			\end{shaded}			
			Extract monthly precipitation values from the Terraclimate ImageCollection
			 through $ ee_extract $. $ ee_extract $ works similar to $ raster::extract $, 
			 you just need to define: the ImageCollection object (x), the geometry (y),
			 and a function to summarize the values (fun).	  
        \begin{shaded}
        	\begin{verbatim}
        		{
        			Precipitation       <- ee_extract(x = Precipitation , y = aoi.ee,
        			sf = FALSE,scale = 250, fun = ee$Reducer$mean(), via = "drive",
        			quiet = T)
        			
        			MinimumTemperature  <- ee_extract(x = MinimumTemperature , 
        			y = aoi.ee, sf = FALSE,scale = 250, fun = ee$Reducer$mean(),
        		 via = "drive", quiet = T)
        			
        			MaximumTemperature  <- ee_extract(x = MaximumTemperature ,
        			y = aoi.ee, sf = FALSE,scale = 250, fun = ee$Reducer$mean(),
        			via = "drive", quiet = T)
        			
        			Evapotranspiration  <- ee_extract(x = Evapotranspiration ,
        			y = aoi.ee, sf = FALSE,scale = 250, fun = ee$Reducer$mean(),
        			via = "drive", quiet = T)
        			
        			Humidity            <- ee_extract(x = Humidity , y = aoi.ee,
        			sf = FALSE, scale = 250,fun = ee$Reducer$mean(), via = "drive",
        			quiet = T)
        			
        			Drought             <- ee_extract(x = Drought , y = aoi.ee, 
        			sf = FALSE, scale = 250,fun = ee$Reducer$mean(), via = "drive",
        			 quiet = T)
        		}
        		
        	\end{verbatim}
        \end{shaded}
    Save the Data to an excell file 
		\begin{shaded}
			\begin{verbatim}
				{
					write.csv(Precipitation,"Data/Precipitation.csv")
					write.csv(MinimumTemperature,"Data/MinimumTemperature.csv")
					write.csv(MaximumTemperature,"Data/MaximumTemperature.csv")
					write.csv(Evapotranspiration,"Data/Evapotranspiration.csv")
					write.csv(Humidity,"Data/Humidity.csv")
					write.csv(Drought,"Data/Drought.csv")
				}
			\end{verbatim}
		\end{shaded}
		Tidy the Data
		\begin{shaded}
			\begin{verbatim}
				{
					Precipitation <- Precipitation%>%
					pivot_longer(starts_with("X20"),names_to =  c("X","Date"),
					names_pattern = "(.)(.+)",values_to = "Precipitation")%>%
					separate(Date,into = c("Date","Pr"),sep = "_")%>% 
					separate(Date, into = c('year', 'month'), sep = -2, convert = TRUE)%>%
					select(year,month,Precipitation)}
				{
					MinimumTemperature <- MinimumTemperature%>%
					pivot_longer(starts_with("X20"),names_to =  c("X","Date"),
					names_pattern = "(.)(.+)",values_to = "MinTemperature")%>%
					separate(Date,into = c("Date","tmmn"),sep = "_")%>% 
					separate(Date, into = c('year', 'month'), sep = -2, convert = TRUE)%>%
					select(year,month,MinTemperature)}
				{MaximumTemperature <- MaximumTemperature%>%
					pivot_longer(starts_with("X20"),names_to =  c("X","Date"),
					names_pattern = "(.)(.+)",values_to = "MaxTemperature")%>%
					separate(Date,into = c("Date","tmmx"),sep = "_")%>% 
					separate(Date, into = c('year', 'month'), sep = -2, convert = TRUE)%>%
					select(year,month,MaxTemperature)}
			\end{verbatim}
		\end{shaded}
	\chapter{Appendix  Chapter 2}
	\begin{shaded}
		\begin{verbatim}
			getQABits <- function(image, qa) {
				# Convert binary (character) to decimal (little endian)
				qa <- sum(2^(which(rev(unlist(strsplit(as.character(qa), "")) == 1))-1))
				# Return a mask band image, giving the qa value.
				image$bitwiseAnd(qa)$lt(1)
			}	mod.clean <- function(img) {
				# Extract the NDVI band
				ndvi_values <- img$select("NDVI")
				# Extract the quality band
				ndvi_qa <- img$select("SummaryQA")
				# Select pixels to mask
				quality_mask <- getQABits(ndvi_qa, "11")
				# Mask pixels with value zero.
				ndvi_values$updateMask(quality_mask)$divide(ee$Image$constant(10000))
				#0.0001 is the MODIS Scale Factor
			}
		   Date <- Sys.Date()			
			modis.evi <- ee$ImageCollection("MODIS/006/MOD13Q1")$filter(ee$Filter$
			date('2000-01-01',rdate_to_eedate(Date)))$map(mod.clean)			
			
			cc.proj <- st_transform(cc, st_crs(2992))
			hex <- st_make_grid(x = cc.proj, cellsize = 17080, square = FALSE) %>%
			st_sf() %>%
			rowid_to_column('hex_id')
			hex <- hex[cc.proj,]
			plot(hex)			
			{
			cc.evi <- ee_extract(x = modis.evi, y = hex["hex_id"], sf = FALSE, scale = 250,
			 fun = ee$Reducer$mean(), via = "drive", quiet = T)
				evi.df <- as.data.frame(cc.evi)
				write.csv(x = evi.df, file = "Data/rgeedf.csv")
			}
			
			cc.evi = evi.df <-read.csv("Data/rgeedf.csv")
			colnames(evi.df) <- c('hex_id', stringr::str_replace_all(substr(colnames
			 (evi.df[, 2:ncol(evi.df)]), 2, 11), "_", "-"))				
			{
				evi.hw.lst <- list() 
				#Create an empty list, this will be used to house the time series
				 projections for each cell. 
				evi.dcmp.lst <- list() 
				#Create an empty list, this will be used to house the time series
				 decomposition for each cell.
				evi.df<-evi.df[,-2]
				evi.trend <- data.frame(hex_id = evi.df$hex_id, na.cnt = NA, na.cnt.2 = NA,
				 trend = NA, p.val = NA, r2 = NA, std.er = NA, trnd.strngth = NA,
				  seas.strngth = NA) 
				#This data frame will hold the trend data
				Dates <- data.frame(date = seq(as.Date('2000-01-01'), Date, "month"))
				Dates$month <- month(Dates$date)
				Dates$year <- year(Dates$date)
				i <- 1
			}
			tsv <- data.frame(evi = t(evi.df[i, 2:ncol(evi.df)])) 
			#converting the data to a transposed data frame
			colnames(tsv) <- c("evi")
			
			head(tsv) #let's take a look			
			
			na.cnt <- length(tsv[is.na(tsv)])
			 #We want to get an idea of the number of entries with no EVI value
			evi.trend$na.cnt[i] <- na.cnt
			td <- tsv %>% 
			mutate(month = month(as.Date(rownames(tsv))), year = year(as.Date(rownames(tsv)))) %>% 
			group_by(year, month) %>%
			summarise(mean_evi = mean(evi, na.rm = T), .groups = "keep") %>%
			as.data.frame()
			head(td)			
			
			td$date <- as.Date(paste0(td$year, "-", td$month, "-01"))
			dx <- Dates[!(Dates$date %in% td$date),]
			dx			
			
			dx$mean_evi <- NA
			tdx <- rbind(td, dx) %>% 
			arrange(date)
			head(tdx)
			write.csv(tdx,"Data/NDVI.csv")
			na.cnt <- length(tdx[is.na(tdx)])
			evi.trend$na.cnt.2[i] <- na.cnt 
			#add count of na values to dataframe
			rm(td, dx)
			 #remove data we're no longer using, this is a good rule of thumb,
			 especially when working with larger datasets.
			tdx <- ts(data = tdx$mean_evi, start = c(2000, 1), end = c(2022, 01), frequency = 12)
			 #convert data to time series.
			plot(tdx,pch = 16, 
			xlab = "Time", ylab = "EVI ", col = "#2E9FDF")			
			
			tdx <- if(na.cnt > 0){imputeTS::na_kalman(tdx, model = "auto.arima", smooth = T)}
			 else {
				tdx
			}
			plot(tdx,pch = 16, frame = FALSE,
			xlab = "Time", ylab = "EVI ", col = "#2E9FDF")			
			
			tdx.dcp <- stl(tdx, s.window = 'periodic')
			plot(tdx.dcp,pch = 16, frame = FALSE, col = "#2E9FDF")			
			
			Tt <- trendcycle(tdx.dcp)
			St <- seasonal(tdx.dcp)
			Rt <- remainder(tdx.dcp)
			plot(Rt)
			plot(Tt)
			plot(St)			
			
			# The Stationary Signal and ACF
			plot(Rt,col= "red", main = "Stationary Signal")
			acf(Rt, lag.max = length(Rt),
			xlab = "lag", ylab = 'ACF', main = '')
			
			#The Trend Signal anf ACF
			
			plot(Tt,col= "red",main = "Trend Signal")
			acf(Tt, lag.max = length(Tt),
			xlab = "lag", ylab = "ACF", main = '')			
			
			tseries::adf.test(tdx)		
			
			tdx.ns <- data.frame(time = c(1:length(tdx)), trend = tdx - tdx.dcp$time.series[,1])
			Tt <- trendcycle(tdx.dcp)
			St <- seasonal(tdx.dcp)
			Rt <- remainder(tdx.dcp)
			trend.summ <- summary(lm(formula = trend ~ time, data = tdx.ns)) #tslm
			plot(tdx.ns,pch = 16, 
			xlab = "Time", ylab = "Trend ", col = "#2E9FDF")
			abline(a = trend.summ$coefficients[1,1], b = trend.summ$coefficients[2,1], col='red')
			
			
			evi.trend$trend[i] <- trend.summ$coefficients[2,1]
			evi.trend$trnd.strngth[i] <- round(max(0,1 - (var(Rt)/var(Tt + Rt))), 1) 
			#Trend Strength Calculation 
			evi.trend$seas.strngth[i] <- round(max(0,1 - (var(Rt)/var(St + Rt))), 1)
			#Seasonal Strength Calculation
			evi.trend$p.val[i] <- trend.summ$coefficients[2,4]
			evi.trend$r2[i] <- trend.summ$r.squared
			evi.trend$std.er[i] <- trend.summ$sigma
			evi.trend[i,]		
			
			plot(evi.hw <- forecast::hw(y = tdx, h = 12, damped = T))
			
			evi.trend <- read.csv("Data/rgeedf.csv")
			
			for(i in 1:nrow(evi.df)){
				tsv <- data.frame(evi = t(evi.df[i, 2:ncol(evi.df)])) 
				colnames(tsv) <- c("evi")
				na.cnt <- length(tsv[is.na(tsv)])
				evi.trend$na.cnt[i] <- na.cnt
				if(na.cnt < 263){
					td <- tsv %>% 
					mutate(month = month(as.Date(rownames(tsv))), year = year(as.Date(rownames(tsv)))) %>%
					group_by(year, month) %>%
					summarise(mean_evi = mean(evi, na.rm = T), .groups = "keep") %>%
					as.data.frame()
					td$date <- as.Date(paste0(td$year, "-", td$month, "-01"))
					dx <- Dates[!(Dates$date %in% td$date),]
					dx$mean_evi <- NA
					tdx <- rbind(td, dx) %>% 
					arrange(date)
					na.cnt <- length(tdx[is.na(tdx)])
					evi.trend$na.cnt.2[i] <- na.cnt
					rm(td, dx)
					tdx <- ts(data = tdx$mean_evi, start = c(2001, 1), end = c(2019, 11), frequency = 12)
					tdx <- if(na.cnt > 0){imputeTS::na_kalman(tdx, model = "auto.arima", smooth = T)}
					 else {
						tdx
					}					
					tdx.dcp <- stl(tdx, s.window = 'periodic')
					evi.dcmp.lst[[i]] <- tdx.dcp
					#This will save our decomposition plots
					plot(tdx.dcp)
					dev.off()
					tdx.ns <- data.frame(time = c(1:length(tdx)), trend = tdx - tdx.dcp$time.series[,1])
					Tt <- trendcycle(tdx.dcp)
					St <- seasonal(tdx.dcp)
					Rt <- remainder(tdx.dcp)
					trend.summ <- summary(lm(formula = trend ~ time, data = tdx.ns)) #tslm
					evi.trend$trend[i] <- trend.summ$coefficients[2,1]
					evi.trend$trnd.strngth[i] <- round(max(0,1 - (var(Rt)/var(Tt + Rt))), 1) 
					
					evi.trend$seas.strngth[i] <- round(max(0,1 - (var(Rt)/var(St + Rt))), 1)
					 #Seasonal Strength Calculation
					evi.trend$p.val[i] <- trend.summ$coefficients[2,4]
					evi.trend$r2[i] <- trend.summ$r.squared
					evi.trend$std.er[i] <- trend.summ$sigma
					evi.hw <- forecast::hw(y = tdx, h = 12, damped = T)
					evi.hw.lst[[i]] <- evi.hw
					# plot(evi.hw)\
					# rm(evi.hw, trend.summ, tdx.ns, tdx.dcp, Tt, St, Rt, tdx, na.cnt)
				} else {
					evi.ts[[i]] <- NA
				}
			}			
			head(evi.trend) #Let's take a peak			
			
			evi.trend$system.index <- cc.evi[,1]
			hex_trend <- hex %>%
			left_join(evi.trend, by = 'hex_id', keep = F) %>%
			replace(is.na(.), 0)
			hex_trend <- st_transform(hex_trend, st_crs(4326))	
			
			**create a Leaflet Web Map!**			
			
			library(classInt)
			trend_brks <- classIntervals(hex_trend$trend, n=11, style = "fisher")
			colorscheme <- RColorBrewer::brewer.pal(n = 11, 'RdYlGn')
			palette_sds <- leaflet::colorBin(colorscheme, domain = hex_trend$trend, 
			bins=trend_brks$brks, na.color = "#ffffff", pretty = T)
			
			pop <- paste0(
			"<b> Hex ID: </b>",hex_trend$hex_id,"<br><b>NA Count:</b>",
			 hex_trend$na.cnt+hex_trend$na.cnt.2,"<br><b>Trend: </b>",
			 format(round(hex_trend$trend, 4), scientific = FALSE),"<br><b> P-Value:</b>",
			 round(hex_trend$p.val, 4),"<br><b>R2: </b>",round(hex_trend$r2, 4),
			 "<br><b>Std Err: </b>",round(hex_trend$std.er, 4),"<br><b>Trend Strength:</b>",
			 round(hex_trend$trnd.strngth, 2),"<br><b>Seasonal Strength:</b>",
			  round(hex_trend$seas.strngth, 4),"<br>"
			  )
			#Here we're creating a popup for our interactive map.			
			
			library(leaflet)
			library(dplyr)
			map <- hex_trend %>%
			leaflet() %>%
			setView(5.96475,-1.782181, 9) %>%
			addProviderTiles("Esri.WorldTopoMap", group = "Topo Map") %>%
			addProviderTiles("Esri.WorldImagery", group = "Imagery", 
			options = providerTileOptions(opacity = 0.7)) %>%
			addPolygons(
			fillColor = ~palette_sds(hex_trend$trend),
			fillOpacity = hex_trend$trnd.strngth,
			opacity = 0.5,
			weight = 0.1,
			color='white', 
			group = "Hexbins", 
			highlightOptions = highlightOptions(
			color = "white",
			weight = 2,
			bringToFront = TRUE),
			popup = pop,
			popupOptions = popupOptions(
			maxHeight = 250, 
			maxWidth = 250)) %>%
			addLegend(
			title = "Trend: lm(EVI ~ Month)",
			pal = palette_sds,
			values = hex_trend$trend,
			opacity = 0.7,
			labFormat = labelFormat(
			digits = 5)) %>%
			addLayersControl(
			baseGroups = c("Topo Map", "Imagery"),
			overlayGroups = c("Hexbins"),
			options = layersControlOptions(collapsed = FALSE)) %>%
			addScaleBar(position='bottomleft')
			
			map
			
		\end{verbatim}
	\end{shaded}
	\section{VAR}
	
			Import Data
			\begin{shaded}
				\begin{verbatim}
			DATA <- read.csv("Data/Data.csv")%>%
			dplyr::select(date,mean_evi,Precipitation,Evapotranspiration,
			MiniTemperature,MaxTemperature,Humidity,Drought)
			head(DATA)
			# Check Missing Values
			
			visdat::vis_dat(DATA)
			skimr::skim_tee(DATA)
			summary(DATA)
				\end{verbatim}
		\end{shaded}
	
				\begin{shaded}
				\begin{verbatim}
			# Correlation			
			colnames(DATA)<- c("Date","EVI","Precipitation","Evapiration","TempMin","TempMax",
			                  "Humidity","Drought")
			plot(corr_coef(DATA))
				\end{verbatim}
		\end{shaded}
	
				\begin{shaded}
				\begin{verbatim}
			# Select a set of predictors with minimal multicollinearity			
			non_collinear_vars(DATA,TempMin,TempMax,Evapiration,Precipitation,Humidity,Drought,
			                    max_vif =3)
			
			# Removed correlated variable			
			DATA <- DATA%>%
			dplyr::select(Date,Drought, TempMin, Precipitation, TempMax, Evapiration,EVI)			
			colnames(DATA)
			head(DATA)
			
				\end{verbatim}
	     	\end{shaded}
	Input Missing Values 
				\begin{shaded}
				\begin{verbatim}
			DATA$EVI <- if(is.na(DATA$EVI) > 0){imputeTS::na_kalman(DATA$EVI, 
				model = "auto.arima",smooth = T)} else {
				DATA$EVI
			}
			head(DATA)
		 	
				\end{verbatim}
		       \end{shaded}
	
			Check if Impute worked
				\begin{shaded}
				\begin{verbatim}
			vis_dat(DATA)
			df<- DATA%>%
			dplyr::select(-Date)
			# 1st differenced data
			# df <- as.data.frame(diff(as.matrix(DATA), lag = 1))
			#========================================================
			# Summary of ADF test of level variables
			#========================================================
			
			adf.none  <- list(
			EVI = ur.df(DATA$EVI, type='none', selectlags = c("AIC")),
			Precipitation = ur.df(DATA$Precipitation, type='none', selectlags = c("AIC")),
			Evapiration = ur.df(DATA$Evapiration, type='none', selectlags = c("AIC")),
			TempMin = ur.df(DATA$TempMin, type='none', selectlags = c("AIC")),
			TempMax = ur.df(DATA$TempMax, type='none', selectlags = c("AIC")),
			Drought = ur.df(DATA$Drought, type='none', selectlags = c("AIC"))
			)
			adf.drift  <- list(
			EVI = ur.df(DATA$EVI, type='drift', selectlags = c("AIC")),
			Precipitation = ur.df(DATA$Precipitation, type='drift', selectlags = c("AIC")),
			Evapiration = ur.df(DATA$Evapiration, type='drift', selectlags = c("AIC")),
			TempMin = ur.df(DATA$TempMin, type='drift', selectlags = c("AIC")),
			TempMax = ur.df(DATA$TempMax, type='drift', selectlags = c("AIC")),
			Drought = ur.df(DATA$Drought, type='drift', selectlags = c("AIC"))
			)
			adf.trend  <- list(
			EVI = ur.df(DATA$EVI, type='trend', selectlags = c("AIC")),
			Precipitation = ur.df(DATA$Precipitation, type='trend', selectlags = c("AIC")),
			Evapiration = ur.df(DATA$Evapiration, type='trend', selectlags = c("AIC")),
			TempMin = ur.df(DATA$TempMin, type='trend', selectlags = c("AIC")),
			TempMax = ur.df(DATA$TempMax, type='trend', selectlags = c("AIC")),
			Drought = ur.df(DATA$Drought, type='trend', selectlags = c("AIC"))
			)
				\end{verbatim}
		\end{shaded}
	
				\begin{shaded}
				\begin{verbatim}			
			summary(adf.none$EVI)
			summary(adf.none$Precipitation)
			summary(adf.none$Evapiration)
			summary(adf.none$TempMin)
			summary(adf.none$TempMax)
			summary(adf.none$Drought)
			
				\end{verbatim}
		\end{shaded}
	
			
				\begin{shaded}
				\begin{verbatim}
			summary(adf.trend$EVI)
			summary(adf.trend$Precipitation)
			summary(adf.trend$Evapiration)
			summary(adf.trend$TempMin)
			summary(adf.trend$TempMax)
			summary(adf.trend$Drought)			
				\end{verbatim}
		\end{shaded}

			
				\begin{shaded}
				\begin{verbatim}
			summary(adf.drift$EVI)
			summary(adf.drift$Precipitation)
			summary(adf.drift$Evapiration)
			summary(adf.drift$TempMin)
			summary(adf.drift$TempMin)
			summary(adf.drift$Drought)
				\end{verbatim}
		\end{shaded}			
			1st differenced data
				\begin{shaded}
				\begin{verbatim}
			df <- as.data.frame(diff(as.matrix(DATA[,-1]), lag = 1))
			df.adf.none  <- list(
			EVI = ur.df(df$EVI, type='none', selectlags = c("AIC")),
			Precipitation = ur.df(df$Precipitation, type='none', selectlags = c("AIC")),
			Evapiration = ur.df(df$Evapiration, type='none', selectlags = c("AIC")),
			TempMin = ur.df(df$TempMin, type='none', selectlags = c("AIC")),
			TempMax = ur.df(df$TempMax, type='none', selectlags = c("AIC")),
			Drought = ur.df(df$Drought, type='none', selectlags = c("AIC"))
			)
			df.adf.drift  <- list(
			EVI = ur.df(df$EVI, type='drift', selectlags = c("AIC")),
			Precipitation = ur.df(df$Precipitation, type='drift', selectlags = c("AIC")),
			Evapiration = ur.df(df$Evapiration, type='drift', selectlags = c("AIC")),
			TempMin = ur.df(df$TempMin, type='drift', selectlags = c("AIC")),
			TempMax = ur.df(df$TempMax, type='drift', selectlags = c("AIC")),
			Drought = ur.df(df$Drought, type='drift', selectlags = c("AIC"))
			)
			df.adf.trend  <- list(
			EVI = ur.df(df$EVI, type='trend', selectlags = c("AIC")),
			Precipitation = ur.df(df$Precipitation, type='trend', selectlags = c("AIC")),
			Evapiration = ur.df(df$Evapiration, type='trend', selectlags = c("AIC")),
			TempMin = ur.df(df$TempMin, type='trend', selectlags = c("AIC")),
			TempMax = ur.df(df$TempMax, type='trend', selectlags = c("AIC")),
			Drought = ur.df(df$Drought, type='trend', selectlags = c("AIC"))
			)
				\end{verbatim}
		\end{shaded}
	
			The ADF result for our variable from the above R code is generated as follows
			
				\begin{shaded}
				\begin{verbatim}
			summary(adf.trend$EVI)
			summary(df.adf.trend$Precipitation)
			summary(df.adf.trend$Evapiration)
			summary(df.adf.trend$TempMin)
			summary(df.adf.trend$TempMax)
			summary(df.adf.trend$Drought)
			
				\end{verbatim}
		\end{shaded}
	
			Interpretation of ADF test follow the general-to-specific approach. As such, three regression models are applied sequentially.
				\begin{shaded}
				\begin{verbatim}
			#========================================================
			# General-to-Specific Investigation
			# The case of EVI variable
			#========================================================
			
			print("Level Variable with Trend")
			cbind(t(df.adf.trend$EVI@teststat), df.adf.trend$EVI@cval)
			
			print("Level Variable with Trend")
			cbind(t(df.adf.trend$Precipitation@teststat), df.adf.trend$Precipitation@cval)
			
			print("Level Variable with Trend")
			cbind(t(df.adf.trend$Evapiration@teststat), df.adf.trend$Evapiration@cval)
			
			print("Level Variable with Trend")
			cbind(t(df.adf.trend$TempMin@teststat), df.adf.trend$TempMin@cval)
			
			print("Level Variable with Trend")
			cbind(t(df.adf.trend$TempMax@teststat),df.adf.trend$TempMax@cval)
			
			print("Level Variable with Trend")
			cbind(t(df.adf.trend$Drought@teststat), df.adf.trend$Drought@cval)
			
			print("1st Diff. Variable with Drift and Trend")
			cbind(t(df.adf.trend$EVI@teststat), df.adf.trend$EVI@cval)
			
			print("1st Diff. Variable with Drift")
			cbind(t(df.adf.drift$EVI@teststat), df.adf.drift$EVI@cval)
			
			print("1st Diff. Variable with None")
			cbind(t(df.adf.none$EVI@teststat), df.adf.none$EVI@cval)
			
				\end{verbatim}
		\end{shaded}
	
			Finally, we can conclude that logarithm of real income contains a unit root and 
			can be stationay time series by differencing the first order. Now that this transformed
			 variable contains no unit root, it can be included in VAR or VECM model.
			
			In this process, the alphanumeric names of test statistics are a little confusing but 
			when we refer the above three specifications of regression equations, the meanings of 
			names of test statistics are clear.
				\begin{shaded}
				\begin{verbatim}
			EVI <- ts(data = df$EVI, start = c(2000, 1), end = c(2022, 01), frequency = 12)
			Precipitation <- ts(data = df$Precipitation, start = c(2000, 1), end = c(2022, 01), frequency = 12)
			Evapiration <- ts(data = df$Evapiration, start = c(2000, 1), end = c(2022, 01), frequency = 12)
			TempMin <- ts(data = df$TempMin, start = c(2000, 1), end = c(2022, 01), frequency = 12)
			TempMax <- ts(data = df$TempMax, start = c(2000, 1), end = c(2022, 01), frequency = 12)
			Drought <- ts(data = df$Drought, start = c(2000, 1), end = c(2022, 01), frequency = 12)
			TimeSeries1 <- cbind(EVI,Precipitation,Evapiration,TempMin,TempMax,Drought)
			plot(TimeSeries1)
				\end{verbatim}
		\end{shaded}
	Time Series using  VAR
				\begin{shaded}
				\begin{verbatim}
			EVI <- ts(data = DATA$EVI, start = c(2000, 1), end = c(2022, 01), frequency = 12)
			Precipitation <- ts(data = DATA$Precipitation, start = c(2000, 1), end = c(2022, 01),
			frequency = 12)
			Evapiration <- ts(data = DATA$Evapiration, start = c(2000, 1), end = c(2022, 01), 
			frequency = 12)
			TempMin <- ts(data = DATA$TempMin, start = c(2000, 1), end = c(2022, 01), frequency = 12)
			TempMax <- ts(data = DATA$TempMax, start = c(2000, 1), end = c(2022, 01), frequency = 12)
			Drought <- ts(data = DATA$Drought, start = c(2000, 1), end = c(2022, 01), frequency = 12)
			TimeSeries <- cbind(EVI,Precipitation,Evapiration,TempMin,TempMax,Drought)
				\end{verbatim}
		\end{shaded}
	
				\begin{shaded}
				\begin{verbatim}
			library(TSstudio)
			ts_plot(EVI)
			ts_plot(Evapiration)
			ts_plot(Drought)
			plot(TimeSeries)
			
				\end{verbatim}
		\end{shaded}
	 Lag Selection \\
	Type of deterministic Regressors to include. We use none because the time series was made stationary using differencing above.
				\begin{shaded}
				\begin{verbatim}
			LagSelection <-VARselect(TimeSeries,type = "const", lag.max = 12) #highest lag order
			LagSelection$selection
			kable(t(LagSelection$criteria))%>%kable_styling(latex_options = c("repeat_header"))                
			
			LagSelection1 <-VARselect(TimeSeries1, 
			type = "const",  
			lag.max = 12) #highest lag order
			LagSelection1$selection
			kable(t(LagSelection1$criteria))%>%kable_styling(latex_options = c("repeat_header")) 
				\end{verbatim}
		\end{shaded}
	
			 Estimating Models
			 
				\begin{shaded}
				\begin{verbatim}
			# Creating a VAR model with vars
			Model <- VAR(TimeSeries,p= 8,lag.max = 12,season = NULL,  exogen = NULL,type = "const")
			summary(Model)
				\end{verbatim}
		\end{shaded}
	
				\begin{shaded}
				\begin{verbatim}
			# Creating a VAR model with vars
			Model1 <- VAR(TimeSeries1,p= 10,lag.max = 12,season = NULL,  exogen = NULL,type = "const")
			summary(Model1)
				\end{verbatim}
		\end{shaded}
	
				\begin{shaded}
				\begin{verbatim}
			predict(Model1, n.ahead = 12, ci = 0.95)
				\end{verbatim}
		\end{shaded}
			forecast is a generic function for forecasting from time series or time series models. The function invokes particular methods which depend on the class of the first argument.
			\begin{shaded}
				\begin{verbatim}
			forecast(Model1)
			plot(forecast(Model1))
				\end{verbatim}
		     \end{shaded}
	
				\begin{shaded}
				\begin{verbatim}
			accuracy(forecast(Model1),d=10, D= 1)
				\end{verbatim}
		      \end{shaded}
	
				\begin{shaded}
				\begin{verbatim}
			library(vars)
			colnames(TimeSeries1) <-c("EVI","Prep","Evap","TMin","TMax","Drght")
			Model2 <- VAR(TimeSeries1,p= 10,lag.max = 12,season = NULL,  exogen = NULL,type = "const")
				\end{verbatim}
		\end{shaded}
		
				\begin{shaded}
				\begin{verbatim}
			plot.varfevd  <-function (x, plot.type = c("multiple", "single"), names = NULL,
			main = NULL, col = NULL, ylim = NULL, ylab = NULL, xlab = NULL,
			legend = NULL, names.arg = NULL, nc, mar = par("mar"), oma = par("oma"),
			addbars = 1, ...)
			{
				K <- length(x)
				ynames <- names(x)
				plot.type <- match.arg(plot.type)
				if (is.null(names)) {
					names <- ynames
				}
				else {names <- as.character(names)
					if (!(all(names %in% ynames))) {
						warning("\nInvalid variable name(s) supplied, using first variable.\n")
						names <- ynames[1]
					}
				}
				nv <- length(names)
				#    op <- par(no.readonly = TRUE)
				ifelse(is.null(main), main <- paste("FEVD for", names), main <- rep(main,
				nv)[1:nv])
				ifelse(is.null(col), col <- gray.colors(K), col <- rep(col,
				K)[1:K])
				ifelse(is.null(ylab), ylab <- rep("Percentage", nv), ylab <- rep(ylab,
				nv)[1:nv])
				ifelse(is.null(xlab), xlab <- rep("Horizon", nv), xlab <- rep(xlab,
				nv)[1:nv])
				ifelse(is.null(ylim), ylim <- c(0, 1), ylim <- ylim)
				ifelse(is.null(legend), legend <- ynames, legend <- legend)
				if (is.null(names.arg))
				names.arg <- c(paste(1:nrow(x[[1]])), rep(NA, addbars))
				plotfevd <- function(x, main, col, ylab, xlab, names.arg,
				ylim, ...) {
					addbars <- as.integer(addbars)
					if (addbars > 0) {
						hmat <- matrix(0, nrow = K, ncol = addbars)
						xvalue <- cbind(t(x), hmat)
						barplot(xvalue, main = main, col = col, ylab = ylab,
						xlab = xlab, names.arg = names.arg, ylim = ylim,
						legend.text = legend, ...)
						abline(h = 0)
					}
					else {xvalue <- t(x)
						barplot(xvalue, main = main, col = col, ylab = ylab,
						xlab = xlab, names.arg = names.arg, ylim = ylim,
						...)
						abline(h = 0)
					}
				}
				if (plot.type == "single") {
					for (i in 1:nv) {
						plotfevd(x = x[[names[i]]], main = main[i], col = col,
						ylab = ylab[i], xlab = xlab[i], names.arg = names.arg,
						ylim = ylim, ...)
					}
				}
				else if (plot.type == "multiple") {
					if (missing(nc)) {
						nc <- ifelse(nv > 4, 2, 1)
					}
					nr <- ceiling(nv/nc)
					par(mfcol = c(nr, nc), mar = mar, oma = oma)
					for (i in 1:nv) {
						plotfevd(x = x[[names[i]]], main = main[i], col = col,
						ylab = ylab[i], xlab = xlab[i], names.arg = names.arg,
						ylim = ylim, ...)
					}
				}
				#    on.exit(par(op))
			}
				\end{verbatim}
		\end{shaded}
	
			\begin{shaded}
				\begin{verbatim}
			
			win.graph(width=13,height=8)
			layout(matrix(1:6,ncol=1))
			plot.varfevd(fevd(Model2, n.ahead = 10 ),plot.type = "multiple", col=1:6)
			
				\end{verbatim}
		\end{shaded}
			 Impulse Responds Analysis
				\begin{shaded}
				\begin{verbatim}
			plot(irf(Model2,impulse = "EVI",response = "EVI"))
			plot(irf(Model2,impulse = "EVI",response = "Prep"))
			plot(irf(Model2,impulse = "EVI",response = "Evap"))
			plot(irf(Model2,impulse = "EVI",response = "TMin"))
			plot(irf(Model2,impulse = "EVI",response = "TMax"))
			plot(irf(Model2,impulse = "EVI",response = "Drght"))
			\end{verbatim}
	    \end{shaded}	
		\begin{shaded}
				\begin{verbatim}
			grangertest(EVI~Precipitation, order = 12, data = TimeSeries1)
			grangertest(EVI~TempMin, order = 12, data = TimeSeries1)
			grangertest(EVI~TempMax, order = 12, data = TimeSeries1)
			grangertest(EVI~Evapiration, order = 12, data = TimeSeries1)
			grangertest(EVI~Drought, order = 12, data = TimeSeries1)
				\end{verbatim}
		\end{shaded}
	
		\begin{shaded}
				\begin{verbatim}
			plot(forecast(Model1))
				\end{verbatim}
		\end{shaded}
			
	
\end{document}